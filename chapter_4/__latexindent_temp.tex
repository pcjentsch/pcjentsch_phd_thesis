\chapter{Conclusion}

\section{Summary of findings}

I have presented three projects that exhibit adaptations of models of infectious dynamics to furthering the understanding of complex systems. The technique used in each chapter was to project the time-evolution of a particular population with age or spatial structure into the language of dynamical systems, and use the tools we have in that realm to hopefully provide insight about the natural system. We usually think of natural systems, such as a population undergoing a pandemic, are a massive  is about projecting down to the correct low-dimensional space, preserving the most significant features. Part of this choice is usually clear, we almost always want to preserve the number of infected people in a disease model, for instance. 


First paper

- we did an age structured differential equation model of covid-19, and we coupled it with imitation dynamics to account for population beliefs about the payoff of social distancing and other NPIs

- 



\section{Future work}



\section{Concluding comments}