
\chapter{Conclusion}

\section{Summary of findings}

I have presented three projects that exhibit extensions of models of complex infectious systems. The technique used in each chapter was to project the time-evolution of a particular population with age or spatial structure into the language of dynamical systems and use the tools we have in that realm to hopefully provide insight about the corresponding natural system. It is common to imagine complex systems in nature, such as a population undergoing a pandemic, as a dynamical system with an enormous but finite number of dimensions. Understanding of such a massive system can be at least partially achieved by projecting down to the low-dimensional space which preserves the most significant features.  

The first chapter presented a model of Sars-CoV-2 spreading throughout a population, coupled to population opinion dynamics on the use of NPIs. The mortality rate after Covid-19 diagnosis varies significantly based on age, therefore our research questions for this model regarded age-based vaccination strategies. We used an age-structured compartmental model to represent the population for this reason. Availability of vaccines, lockdown timing, distancing and the effect of these factors on vaccination planning were explored in the analysis. Age-specific transmission is proportional to empirically-derived location specific contact rates given by Prem et al \cite{prem2020projecting}. The contribution of the contact rates for a given location to the overall contact rate is determined by the fraction of people using NPIs at that point, and the government lockdown policy. The fraction of people using NPIs is also treated as a dynamic variable. The evolution over time of NPI use is determined by a replicator equation where the payoff for using NPIs depends on the current ascertained case level, and the current population sentiment towards NPIs. We fit all parameters, using both case and mobility data from the beginning of the pandemic in Ontario, until November 12th, 2020. A fraction of the vaccines available in a given day are allocated to each age group according to a vaccination strategy. Vaccines leftover from the age groups specified in the strategy are distributed uniformly to the remaining age groups. The four vaccination strategies considered are: oldest first, youngest first, uniform, and contact-based. The contact-based strategy determines proportions based upon the groups most likely to be infected, according the the contact matrices. In contrast, the oldest-first strategy to represent vaccination of groups most vulnerable to Sars-CoV-2 infection. Our model shows that under some conditions, such as a low vaccination supply, and an early vaccination begin date, the contact-based vaccination strategy is most effective at reducing mortality in the long term. Otherwise, it is best to vaccinate the oldest (most vulnerable) populations first.


In the second chapter, we extended a model of forest pest spread via firewood transport of \cite{barlow2014modelling} to a network, and analysed the efficacy of various prevention mechanisms. Compartmental models are designed to represent populations that are approximately well-mixed, that is each member of a compartment has the same statistical properties as any other member. In the case of the forest pest model, we assume that each patch of forest is homogenous, but that these homogenous patches are connected via human transport of trees as firewood. It is common in eastern north america to see individuals selling wood from trees on their property, and this is often more convenient than wood from inside the park area. Research has shown that at least a few invasive insects harmful to forest ecosystems in north america are transported this way \cite{koch2014using, tobin2010does, haack2010incidence}. Barlow et al. \cite{barlow2014modelling} therefore couple the infection dynamics of the forest pest to the social dynamics of firewood transport. Their models coupled only a few patches of forest, and only considered altering the price of firewood as an countermeasure. Our extension of their model to an empirical network of several thousand patches \cite{koch2014using} incorporated other methods of slowing the spread, such as direct interception, broader information campaigns, and patch quarantine, in order to inform policy. Our analysis consisted of evaluating these countermeasures over realistic parameter ranges to determine the conditions under which each is a feasible approach to slow the spread of invasive species. We found that extraordinary measures are needed to demonstrably reduce total attack rates of a pest over 20 years from detection, over most parameter values.

The third chapter of this thesis covered our investigation of a simple fire model coupled to the MPB model of Duncan et al \cite{duncan2015model}. As discussion in the introduction, wildfire is a crucial part of the ecosystem where MPB is native. The host species that MPB prefers most are highly adapted to frequent wildfires, and depend on these disturbances to outcompete other tree species and maintain the large monospecific stands that we observe. The model we present and analyze is a discrete-time compartmental model, where the host population is age structured. A discrete-time model is used because MPB lifecycles can be approximated well as discrete generations. The host population is age-structured because the susceptibility of a tree to MPB increases sharply at a certain DBH (diameter at breast height), which we assume to be achieved once a host tree reaches 50 years old. To match the discrete yearly approximation of forest dynamics, we also use a dynamical model for the yearly fire burn area. We find that, despite the simplicity of the model, the interaction of these two processes arrives at useful insights. We show that wildfire can increase heterogeneity of stand structure such that MPB outbreak sizes are small. Specifically, increase the susceptibility of a stand to stand-clearing fires provides a regular disturbance, which flattens the age structure of the stand. This confirms old observations made by forest ecologists \cite{kaufmann2008status, seidl2016spatial}, but in a very general model. To this end, we show that even small adjustments to the age structure of juvenile trees in a stand can have large effects in increasing the resilience of a stand against MPB outbreaks.     

These chapters are novel in their application of existing human-environment modeling ideas to questions pertinent to disease and forest pest dynamics. As discussed in the introduction, individual participation is critical to the effectiveness of many NPIs against Covid-19. Age-structure vaccine prioritization has been modelled elsewhere \cite{bubar2020model,hoyt2020vaccine}, but not coupled to behavioural dynamics. Chapter \ref{covidmodel} addresses this gap in the literature. Chapter \ref{firewoodmodel} expands upon Barlow et al. \cite{barlow2014modelling} to use a large real world network, and introduces further interventions. Chapter \ref{mpbmodel} sheds light on the function of stand structure in wildfire-MBP dynamics. To our knowledge, it is the first analysis of a mechanistic model  of stand dynamics subject to these coupled disturbances. The human aspect of this system is the form of the control mechanism we suggest. This research addresses the dearth of models coupling wildfire and MPB in a dynamical system, despite the importance of these disturbances.  



\section{Discussion}

Throughout this thesis we have discussed infections in human-environment systems, represented with compartmental systems of differential or difference equations. We use this framework as a way to homogenize attributes of the population for a particular application. Assuming that there is some spatial or age structure on the hosts lets us further subdivide the population. Analysis of disease models can be focused on the dynamics of a particular outbreak, or set of outbreaks, represented by the transient behavior of the underlying dynamical system. These are generally characteristics of the model output immediately after the introduction of a small number of infected hosts into the population, until the outbreak has ended because the infection has reached an equilibrium. Our model of Covid-19 (Chapter \ref{covidmodel}) follows the outbreak transient from the first day with more than 50 cases until mid-november when the manuscript was submitted for publication. Similarly, our model for forest pest transport (Chapter \ref{firewoodmodel}) considers the outbreak transients arising from the introduction of a new invasive forest pest into the Greater Toronto Area, and minimizing the length of these transients with a few methods. In contrast, Chapter \ref{mpbmodel} covered a model of an endemic forest pest. MPB has been a naturally occurring part of the ecosystem it resides in for many thousands of years, and therefore we assume it has reached an equilibrium solution. Analysis for a disease endemic to humans could follow a similar pattern. For instance, Chitnis et al. studied conditions for the stability of endemic malaria \cite{chitnis2006bifurcation}. Even with systems considering an endemic infection, we can look at transients following some perturbation to the system, such as outcomes following vaccination against human papillomavirus \cite{lee2012mathematical}. 

Indirect protection from infection is a consistent theme throughout this thesis. In an otherwise homogenous model, indirect protection is representable through additional age structure (chapters \ref{covidmodel}, \ref{mpbmodel}) or spatial structure (chapter \ref{firewoodmodel}) in the host population of an infectious process. We explicitly discuss indirect protection frequently in the context of disease models and immunization, but it is also present under the concept of heterogeneity in ecological systems. Chapter \ref{mpbmodel} studies the indirect protection created by heterogeneous age structure, and shows that it can be maintained by  wildfire disturbances in a specific forest ecosystem. Chapter \ref{firewoodmodel} discusses indirect protection, in the form of patch quarantine. In that chapter we conclude that adequate indirect protection is difficult to achieve in a model with many perfectly well-mixed pockets (the individual forest patches). Although, it should be noted that patch quarantine in the forest-pest model context is necessarily a much weaker form of indirect protection than, for example, vaccination in a disease model.

In all three chapters, methods for increasing resilience of host populations to infectious agents are compared. Age-based vaccination strategies (chapter \ref{covidmodel}), forest pest mitigation strategies (chapter \ref{firewoodmodel}), and forest thinning protocols  (chapter \ref{mpbmodel}) are evaluated in terms of their host mortality reduction. We test each strategy over a large range of the parameter space to understand when each strategy works and why. Furthermore, strategies are often learned from the structure of the model itself. The contact-based vaccination strategy in chapter \ref{covidmodel}, for instance, is derived from the model for contact patterns. Similarly, the stand thinning protocol (FTP in chapter \ref{mpbmodel}) was created to use features of the stand dynamics present in the MPB-fire model. 


\section{Limitations and future work}

The work in this thesis is focused on models to understand how dynamics act upon major features of a given system. Our model of Covid-19 was designed to provide guidance on the broad strokes of a vaccination response, and since its publication there has been some work confirming our findings in other settings \cite{chen2021age,hogan2021within}. There are still some areas where we could have improved the model, and opportunities for modeling responses to future pandemics. The vaccination response in our paper is defined by a fraction of vaccines allocated to each age group, if there are fewer people in that age group than vaccines available for that day, these vaccines are allocated uniformly over the remaining age groups. A more intuitive way to allocate available vaccines, and the way that this has been implemented in many jurisdictions, would be with a priority list. In Ontario, Canada, for instance, vaccines were made available first to healthcare workers and the very old, and then those with high risk health conditions.  We assume, for simplicity, that the vaccination rate is constant, but vaccination availability usually ramps up as supply chains are developed. Since the results of the paper are dependent on the vaccination rate, a non-constant vaccination rate would improve predictions. An extension of the model that includes vaccine hesitancy is considered in this chapter, where there is room for additional research. As vaccination against Covid-19 comes underway in 2021, there have been many more cases of hesitancy than we initially considered \cite{schwarzinger2021covid,soares2021factors,callaghan2020correlates}. 

Socio-epidemic models, particularly those based upon game theoretic assumptions about belief formation, usually do not account for structural inequalities present in the study population. We refer to inequality as structural when properties inherent to the construction of health institutions, economic systems, social organizations, and governments result in worse outcomes for certain groups of people living within these systems \cite{harris2020civil,rahman2018constructing,royce2018poverty}. The Covid-19 pandemic has magnified and exacerbated many of these structural inequalities \cite{anyane2020racial,yaya2020ethnic,chen2021revealing,chen2020covid,bowleg2020we,tuyisenge2021covid,horse2021structural,wang2020health}. Our model in chapter \ref{covidmodel} assumes that NPI usage is based on a combination of state policy and individual perception of the severity of the pandemic. While NPI protocols in reality are implemented jurisdiction-wide, actually following these protocols is often a privilege for the wealthy and white for reasons such as access to transportation, housing conditions, and food insecurity \cite{jay2020neighbourhood,mamelund2021social}. While we incorporate essential work in chapter \ref{covidmodel}, we do not incorporate the fact that essential workers are overwhelmingly marginalized groups \cite{lancet2020plight}. In Ontario, paid sick leave was only granted to workers on April 29th, 2021, over one year after the beginning of the pandemic \cite{ontariosickleave}. Behavioral models of vaccination hesitancy are subject to similar criticism. For marginalized groups, there are reasons to distrust a vaccine that should not be described as defecting from the co-operative strategy. Among others, black and indigenous people in North America have survived many centuries of medical experimentation by european colonists \cite{pacheco2013moving,washington2006medical} which could contribute to vaccine hesitancy \cite{jamison2019you,bogart2021covid}. A lack of vaccine uptake could also be due to the significant inequities in vaccine distribution, which we ignore by focusing on vaccine hesitancy \cite{iveniuk2021uneven,osama2021covid,corbie2021vaccine}. In short, game theoretic models generally assume that decisions are based firmly in individual choice, an assumption which does not accurately describe the social and material landscape many people exist within. Nevertheless, population behaviour is an important and often neglected aspect of disease spread. Further research should explore frameworks of behavioural modeling that are able to incorporate these issues. 

Chapter \ref{firewoodmodel} attempts to generally address invasive forest pests in Ontario, Manitoba and Quebec in a network of well-mixed patches of forest. The model in this section uses available traffic data to parameterize the relative magnitude of pest spread between patches, but data on spatial spread of forest pests would be greatly beneficial to calibrate the model for particular species. The complexity and scale of the patch quarantine in the model could be increased further. Our model considered closure of a set number of patches for a fixed time, where patches were closed based upon network centrality. More complex methods could be implemented where patches are closed partially and reactively, based on detected pest locations. Given more data, we could use information about specific pests to better inform patch closure. Beliefs about firewood transport are probably less susceptible to the limitations  of behavioural models mentioned in the previous paragraph. Our assumption that each patch has its own independent beliefs about firewood transport was not explored in the chapter. It is likely that the behavioural dynamics are strongly coupled, and that urban centres are more susceptible to messaging about the dangers of firewood transport than rural areas. The assumption of well-mixed patches aligning with campsites could also be relaxed. One approach would be an agent-based model of the real forested patches in eastern Canada overtop the same campsite network. This method adds orders of magnitude to the dimensionality of the problem, which would be difficult to understand without additional data on pest spread.  

The core idea of chapter \ref{mpbmodel} is that age structure in ecosystems, and the way that disturbances interact within an age structured population, is a useful perspective in understanding them. Rather than adding complexity to our model of MPB and wildfire, an approach that has been explored in many detailed agent-based simulations \cite{caldwell2013simulated,perrakis2014modeling,ager2007modeling,loehman2017interactions}, it would be interesting to explore similar disturbances which might have interactions on a structured population. For example, periodical cicada populations exhibit different predators depending on their life stage \cite{lloyd1966periodical}, which can be longer than the life stage of their predators. 

\section{Concluding comments}

We presented three models aiming to understand the effects of infectious agents within complex human-environment systems. These approaches are novel in their treatment of coupled interactions between these infectious agents, and other significant aspects of their respective systems. Through their application we have gained new insight into the dynamics of such systems, provided actionable policy recommendations, and confirmed patterns observed in empirical research. We hope that future work will be able to extend our methodology to address new problems in these areas.