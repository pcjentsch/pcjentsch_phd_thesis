\chapter{Conclusion}

\section{Summary of findings}

I have presented three projects that exhibit adaptations of models of infectious dynamics to furthering the understanding of complex systems. The technique used in each chapter was to project the time-evolution of a particular population with age or spatial structure into the language of dynamical systems, and use the tools we have in that realm to hopefully provide insight about the natural system. It is common to imagine complex systems in nature, such as a population undergoing a pandemic, as a dynamical system with an enormous but finit number of dimensions. Understanding this system is about projecting down to the low-dimensional space which preserves the most significant features. 

The first chapter presented a model of Sars-CoV-2 spreading throughout a population, coupled to population opinion dynamics on the use of NPIs. Since the mortality rate after Covid-19 diagnosis varies so significantly based on age, our research questions for this model regarded age-based vaccination strategies, and how they were affected by availability and other factors. Therefore, we used an age-structured compartmental model to represent the population. We chose impulsive dynamics for the vaccination process because vaccination is generally not a continuous process, vaccines are administered to a fixed number of people during each day, although it can be (and usually is) approximated as one in continuous compartmental models. We also included a seasonality term, just by varying the infection rate throughout the year. We fit all the parameters, including this seasonality rate, using data from the beginning of the pandemic, until November 12th 2020. 

In the second chapter, we extended a model of forest pest spread via firewood transport of \cite{barlow2014modelling} to a network, and analysed the efficacy of various prevention mechanisms. Compartmental models are designed to represent populations that are approximately well-mixed, that is each member of a compartment has the same statistical properties as any other member. In the case of the forest pest model, we assume that each patch of forest is homogenous, but that these homogenous patches are connected via human transport of trees as firewood. It is common in eastern north america to see individuals selling wood from trees on their property, and this is often more convenient than wood from inside the park area. Research has shown that at least a few invasive insects harmful to forest ecosystems in north america are transported this way \cite{koch2014using, tobin2010does, haack2010incidence}. Barlow et al. \cite{barlow2014modelling} therefore couple the infection dynamics of the forest pest to the social dynamics of firewood transport. Their models coupled only a few patches of forest, and only considered altering the price of firewood as an countermeasure. Our extension of their model to an empirical network of several thousand patches \cite{koch2014using} incorporated other methods of slowing the spread, such as direct interception, broader information campaigns, and patch quarantine, in order to inform policy. Our analysis consisted of evaluating these countermeasures over realistic parameter ranges to determine the conditions under which each is a feasible approach to slow the spread of invasive species. We found that extraordinary measures are needed to demonstrably reduce total attack rates of a pest over 20 years from detection, over most parameter values.

The third chapter of this thesis covered our investigation of a simple fire model coupled to the MPB model of Duncan et al \cite{duncan2015model}. As discussion in the introduction, wildfire is a crucial part of the ecosystem where MPB is native. The host species that MPB prefers most are high adapted to frequent wildfires, and depend on these disturbances to outcompete other tree species and maintain the large monospecific stands that we observe. The model we present and analyze is a discrete-time compartmental model, where the host population is age structured. A discrete-time model is used because MPB lifecycles can be approximated well as discrete generations. The host population is age-structured because the susceptibility of a tree to MPB increases sharply at a certain DBH (diameter at breast height), which we assume to be achieved once a host tree reaches 50 years old. To match the discrete, yearly approximation of forest dynamics, we also use a dynamical model for the yearly fire burn area. We find that, despite the simplicity of the model, the interaction of these two processes arrives at useful insights. We show that wildfire can increase heterogeneity of stand structure such that MPB outbreak sizes are small. Specifically, increase the susceptiblity of a stand to stand-clearing fires provides a regular disturbance, which flattens the age structure of the stand. This confirms old observations made by forest ecologists \cite{kaufmann2008status, seidl2016spatial}, but in a very general model. To this end, we show that even small adjustments to the age structure of juvenile trees in a stand can have large effects in increasing the resilience of a stand against MPB outbreaks.     


\section{Discussion}

Throughout this thesis we have discussed disease dynamics and forest pest models, represented with compartmental systems of differential or difference equations. We use this framework as a way to homogenize attributes of the population for a particular application. Analysis of disease models can be focused on the dynamics of a particular outbreak, or set of outbreaks, represented by the transient behaviour of the underlying dynamical system. These are generally characteristics of the model output immediately after the introduction of a small number of infected hosts into the population, until the outbreak has ended because the infection has reached an equilibrium. Our model of Covid-19 (Chapter \ref{ch1}) follows the outbreak transient from the first day with more than 50 cases until mid-november when the manuscript was submitted for publication. Similarly, our model for forest pest transport (Chapter \ref{ch2}) considers the outbreak transients arising from the introduction of a new invasive forest pest into the GTA, and minimizing the length of these transients with a few methods. In contrast, Chapter 3 covered a model of an endemic forest pest. MPB has been a naturally occuring part of the ecosystem it resides in for many thousands of years, and therefore we assume it has reached an equilbrium solution. Analysis for a disease endemic to humans could follow a similar pattern. For instance, Chitnis et al. studied conditions for the stability of endemic malaria \cite{chitnis2006bifurcation}. Even with systems considering an endemic infection, we can look at transients following some perturbation to the system, such as outcomes following vaccination against human papillomavirus \cite{lee2012mathematical}. 


\section{Limitations and future work}

The work in this thesis is focused on simple models to understand the dependence of dynamics upon major features in the system. Our model of Covid-19 was designed to provide guidance on the broad strokes of a vaccination response. In hindsight, there are many places we could have improved the model, and opportunities for modeling responses to future pandemics. The vaccination response in our paper is defined by a fraction of vaccines allocated to each age group, if there are fewer people in that age group than vaccines available for that day, these vaccines are allocated uniformly over the remaining age groups. A more intuitive way to allocate available vaccines, and the way that this has been implemented in many jurdistictions, would be with a priority list. In Ontario, Canada, for instance, vaccines were made available first to healthcare workers and the very old, and then those with high risk health conditions.  We assume, for simplicity, that the vaccination rate is constant, but vaccination availability usually ramps up as supply chains are developed. Since the results of the paper are dependent on the vaccination rate, it would be interesting to see how a non-constant vaccination rate affects strategy. As vaccination against Covid-19 comes underway in 2021, there have been many cases of vaccine refusal \cite{schwarzinger2021covid,soares2021factors,callaghan2020correlates}. In chapter \ref{ch1}, we consider an extension of the model that also incorporates vaccine refusal, but as the pandemic progresses, we now know that vaccine refusal is a much more important part of the dynamics than we initially suspected it would be, so extensions could be done to focus on that aspect. Since chapter \ref{ch1} was published there has been some work confirming our findings in other settings, \cite{chen2021age,hogan2021within}.

Replicator equations are used for both the physical distancing behavioural dynamics, and the vaccine refusal dynamics in chapter \ref{ch1}. The replicator equations depend on a payoff function for each individual that is difficult to estimate, likely non-autonomous and heterogenous in space. In chapter \ref{ch2}, we used a replicator equation for each patch in the network to help compensate for this dynamic, but this could also be extended to the model in chapter \ref{ch1}. A similar approach has been used in an agent-based setting for disease dynamics \cite{fair2021population}. Imitation dynamics can be extended beyond the prisoners dilemma game in these models, to complex games with many competing strategies \cite{hofbauer1998evolutionary}, which would provide much richer behavioural mechanics.

The core idea of chapter \ref{ch3} is that age structure in ecosystems, and the way that disturbances interact within an age structured population, is a useful perspective in understanding them. Rather than adding complexity to our model of MPB and wildfire, an approach that has been explored in many detailed agent-based simulations \cite{caldwell2013simulated,perrakis2014modeling,ager2007modeling,loehman2017interactions}, it would be interesting to explore similar disturbances which might have interactions on a structured population. For example, periodical cicada populations exhibit different predators depending on their life stage \cite{lloyd1966periodical}, which can be longer than the life stage of their predators. 

\section{Concluding comments}

We presented three models aiming to understand the effects of infectious agents within complex human-environment systems. These approaches are novel in their treatment of coupled interactions between these infectious agents, and other significant aspects of their respective systems. Through their application we have gained new insight into the dynamics of such systems, provided actionable policy recommendations, and confirmed patterns observed in empirical research. We hope that future work will be able to extend our methodology to address new problems in these areas.