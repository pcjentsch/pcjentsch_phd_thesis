\chapter{Conclusion}

\section{Summary of findings}

I have presented three projects that exhibit adaptations of models of infectious dynamics to furthering the understanding of complex systems. The technique used in each chapter was to project the time-evolution of a particular population with age or spatial structure into the language of dynamical systems, and use the tools we have in that realm to hopefully provide insight about the natural system. It is common to imagine complex systems in nature, such as a population undergoing a pandemic, as a dynamical system with an enormous but finit number of dimensions. Understanding this system is about projecting down to the low-dimensional space which preserves the most significant features. 

The first chapter presented a model of Sars-CoV-2 spreading throughout a population, coupled to population opinion dynamics on the use of NPIs. Since the mortality rate after Covid-19 diagnosis varies so significantly based on age, our research questions for this model regarded age-based vaccination strategies, and how they were affected by availability and other factors. Therefore, we used an age-structured compartmental model to represent the population. We chose impulsive dynamics for the vaccination process because vaccination is generally not a continuous process, vaccines are administered to a fixed number of people during each day, although it can be (and usually is) approximated as one in continuous compartmental models. We also included a seasonality term, just by varying the infection rate throughout the year. We fit all the parameters, including this seasonality rate, using data from the beginning of the pandemic, until November 12th 2020. 

In the second chapter, we extended a model of forest pest spread via firewood transport of \cite{barlow2014modelling} to a network, and analysed the efficacy of various prevention mechanisms. Compartmental models are designed to represent populations that are approximately well-mixed, that is each member of a compartment has the same statistical properties as any other member. In the case of the forest pest model, we assume that each patch of forest is homogenous, but that these homogenous patches are connected via human transport of trees as firewood. It is common in eastern north america to see individuals selling wood from trees on their property, and this is often more convenient than wood from inside the park area. Research has shown that at least a few invasive insects harmful to forest ecosystems in north america are transported this way \cite{koch2014using, tobin2010does, haack2010incidence}. Barlow et al. \cite{barlow2014modelling} and Ali et al \cite{ali2015coupled} therefore couple the infection dynamics of the forest pest to the social dynamics of firewood transport. However, their models only consisted of a few patches, and only considered the price of firewood as an countermeasure. Our extension of their model to an empirical network of several thousand patches \cite{koch2014using} incorporated other methods of slowing the spread, such as direct interception, broader information campaigns, and patch quarantine, in order to inform policy. Our analysis consisted of evaluating these countermeasures over realistic parameter ranges to determine the conditions under which each is a feasible approach to slow the spread of invasive species. We found that extraordinary measures are needed to demonstrably reduce total attack rates of a pest over 20 years from detection, over most parameter values.

The third chapter of this thesis covered our investigation of a simple fire model coupled to the MPB model of Duncan et al \cite{duncan2015model}. As discussion in the introduction, wildfire is a crucial part of the ecosystem where MPB is native. The host species that MPB prefers most are high adapted to frequent wildfires, and depend on these disturbances to outcompete other tree species and maintain the large monospecific stands that we observe. The model we present and analyze is a discrete-time compartmental model, where the host population is age structured. A discrete-time model is used because MPB lifecycles can be approximated well as discrete generations. The host population is age-structured because the susceptibility of a tree to MPB increases sharply at a certain DBH (diameter at breast height), which we assume to be achieved once a host tree reaches 50 years old. To match the modes discrete, yearly approximation of forest dynamics, we also use a dynamical model for the yearly fire burn area. We find that, despite the simplicity of the model, the interaction of these two processes arrives at useful insights. We show that wildfire can increase heterogeneity of stand structure such that MPB outbreak sizes are small. Specifically, increase the susceptiblity of a stand to stand-clearing fires provides a regular disturbance, which flattens the age structure of the stand. This confirms old observations made by forest ecologists \cite{kaufmann2008status, seidl2016spatial}, but in a very general model. To this end, we show that even small adjustments to the age structure of juvenile trees in a stand can have large effects in increasing the resilience of a stand against MPB outbreaks.     





\section{Future work}



\section{Concluding comments}