% T I T L E   P A G E
% -------------------
% Last updated October 23, 2020, by Stephen Carr, IST-Client Services
% The title page is counted as page `i' but we need to suppress the
% page number. Also, we don't want any headers or footers.
\pagestyle{empty}
\pagenumbering{roman}

% The contents of the title page are specified in the "titlepage"
% environment.
\begin{titlepage}
        \begin{center}
        \vspace*{1.0cm}

        \Huge
        {\bf Coupled models of structured contagion processes in human-environment systems}

        \vspace*{1.0cm}

        \normalsize
        by \\

        \vspace*{1.0cm}

        \Large
        Peter C. Jentsch \\

        \vspace*{3.0cm}

        \normalsize
        A thesis \\
        presented to the University of Waterloo \\ 
        in fulfillment of the \\
        thesis requirement for the degree of \\
        Doctor of Philosophy \\
        in \\
        Applied Mathematics \\

        \vspace*{2.0cm}

        Waterloo, Ontario, Canada, 2021 \\

        \vspace*{1.0cm}

        \copyright\ Peter C. Jentsch \\
        \end{center}
\end{titlepage}

% The rest of the front pages should contain no headers and be numbered using Roman numerals starting with `ii'
\pagestyle{plain}
\setcounter{page}{2}

\cleardoublepage % Ends the current page and causes all figures and tables that have so far appeared in the input to be printed.
% In a two-sided printing style, it also makes the next page a right-hand (odd-numbered) page, producing a blank page if necessary.

 
% E X A M I N I N G   C O M M I T T E E (Required for Ph.D. theses only)
% Remove or comment out the lines below to remove this page
\begin{center}\textbf{Examining Committee Membership}\end{center}
  \noindent
The following served on the Examining Committee for this thesis. The decision of the Examining Committee is by majority vote.
  \bigskip
  
  \noindent
\begin{tabbing}
Internal-External Member:  \=  \kill % using longest text to define tab length
External Examiner: \> Tim Reluga  \\ 
\> Professor, Mathematics and Biology,\\
\> Pennsylvania State University\\
\end{tabbing} 
  \bigskip
  
  \noindent
\begin{tabbing}
Internal-External Member: \=  \kill % using longest text to define tab length
Supervisors: \> Chris T. Bauch \\
\> Professor, Department of Applied Mathematics, University of Waterloo \\
\>\\
\> Madhur Anand \\
\> Professor, School of Environmental Sciences, University of Guelph \\
\end{tabbing}
  \bigskip
  
  \noindent
  \begin{tabbing}
Internal-External Member: \=  \kill % using longest text to define tab length
Internal Members: \> Sue Ann Campbell \\
\> Professor, Department of Applied Mathematics, University of Waterloo \\ 
\> Zoran Miskovic  \\
\> Professor, Department of Applied Mathematics, University of Waterloo \\
\end{tabbing}
  \bigskip
  
  \noindent
\begin{tabbing}
Internal-External Member: \=  \kill % using longest text to define tab length
Internal-External Member: \> Paul Fieguth \\
\> Professor, Department of Systems Design Engineering,\\
\> University of Waterloo \\
\end{tabbing}
  

\cleardoublepage

% D E C L A R A T I O N   P A G E
% -------------------------------
  % The following is a sample Delaration Page as provided by the GSO
  % December 13th, 2006.  It is designed for an electronic thesis.
 \begin{center}\textbf{Author's Declaration}\end{center}
  
 \noindent
 This thesis consists of material all of which I authored or co-authored: see Statement of Contributions included in the thesis. This is a true copy of the thesis, including any required final revisions, as accepted by my examiners.\\
 I understand that my thesis may be made electronically available to the public.
\cleardoublepage

% Contributions
% -------------------------------
  % The following is a sample Delaration Page as provided by the GSO
  % December 13th, 2006.  It is designed for an electronic thesis.
  \begin{center}\textbf{Statement of Contributions}\end{center}
  
  \begin{itemize}
    
   \item Chapter \ref{covidmodel}: MA and CTB conceptualized the study. All authors designed the model. PCJ developed and analysed the model and generated figures. PCJ and CTB wrote the first draft of the manuscript and accessed and verified the data. All authors revised the manuscript critically for important intellectual content, and read and approved the final version of the manuscript. All authors had full access to all the data in the study, and the corresponding author had final responsibility for the decision to submit for publication. The contents of this chapter are based on the corresponding article published in \textit{Lancet Infectious Diseases} \cite{jentsch2021prioritising}.
   
   \item Chapter \ref{firewoodmodel}: Conceptualization by DY, MA, data curation by PCJ, DY, formal analysis by PCJ, funding acquisition by CTB, DY, MA, investigation by PCJ, CTB, DY, MA. Methodology by PCJ, CTB, DY, MA. Project administration by PCJ, CTB, DY, MA. Resources by CTB, DY, MA. Software written by PCJ. Supervision by CTB, DY, MA, validation by PCJ, CTB, DY, MA, visualization by PCJ, CTB. Original draft written by PCJ. Review & editing by PCJ, CTB, DY, MA. The contents of this chapter are based on the corresponding article published in \textit{PLoS One} \cite{jentsch2020go}.

   
   \item Chapter \ref{mpbmodel}: The work in this chapter is based upon a manuscript accepted for publication in \textit{Theoretical Ecology}. All authors conceived ideas for the study. PCJ designed and coded the model, performed
   analyses, created figures, and drafted the manuscript. All authors revised the manuscript.
   

  \end{itemize}
 \cleardoublepage
 
% A B S T R A C T
% ---------------

\begin{center}\textbf{Abstract}\end{center}

% In 2021, the biosphere of the earth has been almost entirely eradicated and replaced by a thin facsimile. Many of the ecosystems that once spanned the globe have been reduced to skeletons, supported by the scaffolding of dwindling state environmental agency budgets. 

Models of infectious processes are a common feature in the landscape of applied mathematics. It is rare that these processes are isolated from other significant dynamics in nature, and therefore we can incorporate some of the complexity inherent in real systems by coupling infections to major features of the ecosystems they inhabit. Infectious processes can take many forms, but in this thesis we consider three: the COVID-19 pandemic, the invasion of eastern North American forests by wood-borne pests, and the outbreak cycles of an endemic forest pest. The first chapter covers a model of Sars-CoV-2 in a structured population, coupled with a replicator equation representing sentiment towards the use of non-pharmaceutical interventions. We use this human-environment model of to compare the efficacy of vulnerable-first and transmission-preventing age structured vaccination strategies. The buildup of natural immunity in a population combined with a low vaccination supply is shown to cause a transmission-preventing vaccination strategy to be more effective. The second chapter considers a spatially structured model of forest pest contagion over an empirically-derived network of forest patches in eastern Canada. Since these pests can frequently be spread long distances by wood transport, we couple this model to the sentiment of local populations towards avoiding firewood transport from outside their area. Three possible countermeasures to the spread of the invasive pest are compared: social incentives, direct interception of infested firewood, and quarantine of patches. The level of effort needed to significantly reduce forest damage with any of these methods is substantial and unlikely to be implemented. The final chapter extends a model of mountain pine beetle (MPB) in western north american pine forests to incorporate tree mortality due to wildfire. We find that wildfire acts as a disturbance that increases the heterogeneity in age structure, and therefore is able to increase the resilience of the forest to outbreaks of MPB. A targeted thinning procedure aimed specifically at increasing heterogeneity in the forest age structure is proposed and shown to be highly effective at reducing the severity of outbreak. The effectiveness of targeted thinning in the manner described further emphasizes the importance of heterogeneity in forest stand structure. Each model tests the importance of indirect protection in preventing the spread of an infectious agent through a specific host population, with respect to key parameters. Models let us use counterfactuals to gain potentially invaluable understanding of these complex human-environment systems.



\cleardoublepage

% A C K N O W L E D G E M E N T S
% -------------------------------

\begin{center}\textbf{Acknowledgements}\end{center}

I would like to thank the following people for making this thesis possible.

\begin{itemize}
\item My supervisors, Dr. Chris Bauch and Dr. Madhur Anand, for their endless support, understanding, and insight throughout my graduate and undergraduate career.  
  
\item Dr. Sue Ann Campbell, Dr. Zoran Miskovic, Dr. Tim Reluga, and Dr. Paul Fieguth for being on my PhD committee.

\item Dr. Denys Yemshanov for his patience and mentorship on things forest-related. 

\item Dr. Mark Penney for the great conversations, collaboration, and opportunity to write an agent-based model in the last six months of my PhD.

\item Dr. Chrystopher Nehaniv for the opportunities to think and present about topics almost completely different from those in this thesis. 

\item My friends in the Bauch Lab, the applied mathematics department, and outside the university for the time spent not working, and many victorious bar trivia nights.
 
\item Samantha Landry and Emylee Todd for reading and editing the early drafts of this thesis.

\item My mom, dad, and siblings, for their support, encouragement, and not asking "so when are you gonna graduate" too often.  
\end{itemize}

 
\cleardoublepage

% D E D I C A T I O N
% -------------------

\begin{center}\textbf{Dedication}\end{center}

For Emmy, Max, Sam, and the friends that have helped me get through the past 5 years.


\cleardoublepage

% T A B L E   O F   C O N T E N T S
% ---------------------------------
\renewcommand\contentsname{Table of Contents}
\tableofcontents
\cleardoublepage
\phantomsection    % allows hyperref to link to the correct page

% L I S T   O F   F I G U R E S
% -----------------------------
\addcontentsline{toc}{chapter}{List of Figures}
\listoffigures
\cleardoublepage
\phantomsection		% allows hyperref to link to the correct page

% L I S T   O F   T A B L E S
% ---------------------------
\addcontentsline{toc}{chapter}{List of Tables}
\listoftables
\cleardoublepage
\phantomsection		% allows hyperref to link to the correct page

% Change page numbering back to Arabic numerals
\pagenumbering{arabic}

