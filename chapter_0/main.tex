\chapter{Introduction}

I am writing this introduction sitting in my living room in Kitchener, Ontario, staring at the dead ash tree which overlooks the empty lots bordering my home. Ash (\textit{Fraxinus} sp.) were a common group of trees planted along city streets in the eastern Unites States and Canada until the introduction of the emerald ash borer (\textit{Agrilus planipennis}, EAB) to North America in the 1990s \cite{nrcanmpb}. In the past two decades, the emerald ash borer (\textit{Agrilus planipennis}, EAB) has spread throughout the region, killing about 99\% of the ash trees in the regions they invade \cite{nrcaneab,herms2014emerald}. Once a prominent feature of deciduous woodlands in this area, \textit{Fraxinus} is now limited to rare pockets that have escaped the insect, and seedlings too small to be infested. This narrative is a familar one. The American chestnut (\textit{Castanea dentata}) was once a major part of the carolinian forests, an important source of lumber, and food. It was almost completely wiped out as the chestnut blight (\textit{Cryphonectria parasitica}) spread throughout eastern North America in the 19th and 20th centuries. Infectious agents in this way shape the landscapes we inhabit and the ecosytems we exist within. Of course, infectious agents are not limited to arboreal hosts: I have been in my living room staring at this dead ash tree for the past year, sheltering from the global Covid-19 pandemic. 

The various waves of the Black Death, the 1918 influenza pandemic, the HIV/AIDS pandemic, and the current Covid-19 pandemic, have irreversibly shaped our culture. Endemic infectious disease was a massive driving force in the formation of human societies everywhere until very recently. Only in the past century have some parts of the world been able to escape the spectre of endemic diseases such as malaria, polio, influenza, and measles, primarily through the invention of vaccination and understanding of disease spread dynamics. John Snow is considered to be one of the first epidemiologists, for his study of the spatial distribution of London Cholera outbreaks \cite{snow1855mode, brauer2019mathematical}, but the tools used in the field have evolved considerably since that time. Compartmental models have been one of the main tools used to forecast outcomes and mitigation stragies during recent outbreaks of infectious agents \cite{brauer2008compartmental}. Due to the ongoing pandemic, the general public is probably more aware than ever before of these models.

Compartmental models for the spread of infectious diseases are usually considered to have been introduced to the field by public health researchers, such as Hamer, Kermack, and McKendrick \cite{hamer1906epidemic, kermack1927contribution, brauer2019mathematical,edelstein2005mathematical}. This class of models divides the population into homogenous compartments, and describe the rate of movement between these compartments. The quintessential compartmental model is the SIR model (equations \ref{SIR}), so-called for its division of a population into susceptible (the $S(t)$ variable), infected (the $I(t)$ variable), and recovered (the $R(t)$ variable) compartments.

\begin{eqnarray}
    \frac{dS}{dt} = -\beta S I  \\
    \frac{dI}{dt} = \beta S I - \gamma I\\
    \frac{dR}{dt} = \gamma I\\
    \label{SIR}
\end{eqnarray}

The SIR model describes a population undergoing an infection that confers complete immunity after one has been infected, under a great number of simplifying assumptions: each compartment is totally homogenous, the likelihood of infection is proportional to the number of individuals in each compartment, and the probability that an individual recovers is constant per unit time. From the initial work on these models, the invention of computers and numerical integration methods have enabled researchers to get useful results from even the most complex elaborations on this theme. The assumption of homogeneity is a major drawback of the SIR model, so a common extension to this model is to add more structure, usually through further subdivision of the compartments. 

\section{Sars-Cov-2}

Compartmental models have been invaluable in modeling the spread of Sars-CoV-2 \cite{thompson2020epidemiological}. It is difficult to overstate the damage and loss of life that the ongoing COVID-19 pandemic has caused or exacerbated in the past year \cite{miller2020disease,who2021impact}. The first human case of Sars-CoV-2 occurred in late 2019 in central China, almost certainly transmitted from an animal host, very likely a bat \cite{andersen2020proximal,rasmussen2021origins,zhu2020novel}. The virus soon spread throughout the world, and was declared a pandemic by the World Health Organization (WHO), on March 11th, 2020 \cite{who2020announces}. In the intervening months, people around the world endured various levels of non-pharmeceutical interventions (NPIs), such as compulsory face masks, distancing protocols, stay-at-home orders, and workplace closures. Government policy on NPIs has ranged from comprehensive quarantine procedures (in e.g. New Zealand, South Korea, Singapore, Vietnam) to almost nothing at all (United States, Sweden). In the outcomes of the aforementioned countries, empirical research, and modelling studies, NPIs have been shown to be an effective method for the control of Covid-19 \cite{anderson2020estimating,flaxman2020estimating,ferguson2020report,demirguc2020sooner}. 



\section{Imitation dynamics}


The demonstrated effectiveness of NPIs in some countries implies that the immense morbidity and mortality over the past year is not simply a natural disaster, but a humanitarian one. Since our survival depends on our ability to construct a world in which people are incentivized to centre the well-being of others, any attempt to model human outcomes of the pandemic should also attempt to model the incentive strucutures we operate within. Game theory provides a simple, but effective framework for many of the aforementioned systems \cite{andrews2015disease,jentsch2018spatial}. In a game theoretical sense, NPI usage can be viewed as a prisoners dilemma in that everyone either cooperates or defects with the practice of using NPIs, and the decision to defect or to cooperate is based on a combination of the percieved payoff to do so, and the influence of the rest of the population. Table \ref{prisonersdilemma} shows the payoff matrix of this 2-player game. Of course in real life, we are all playing this game, all the time, with everyone. 
\begin{figure}
    \begin{tabular}{ |c|c| c| } \hline
        \diagbox[width = 7em, height = 2em]{P1}{P2} &use NPI& don't use NPI   \\ \hline
        use NPI & \diagbox[width = 13em, height = 8em]{low risk,\\ NPIs unpleasant}{low risk,\\ NPIs unpleasant} &  \diagbox[width = 13em, height = 8em]{med risk,\\ NPIs unpleasant} {med risk}\\ \hline 
        don't use NPI & \diagbox[width = 13em, height = 8em]{med risk}{med risk,\\ NPIs unpleasant} &  \diagbox[width = 13em, height = 8em]{high risk}{high risk}   \\ \hline
    \end{tabular}
    \caption{NPI adoption as a two-player game (between P1 and P2)}
    \label{prisonersdilemma}
\end{figure}


The simplest way to approximate the time evolution of a such a game is with the one-dimensional replicator equation, which approximates these dynamics in terms of the population average \cite{hofbauer1998evolutionary}. Specifically, we introduce a variable $x(t)$ which represents the fraction of people adopting a strategy, then the replicator equation \ref{replicator} gives the time-evolution of $x(t)$ in terms of the payoff for cooperating over defecting, $p(x,t)$. We see immediately that this equation, disregarding $p(x,t) = 0$, has two steady states: $x = 1$ and $x = 0$. Given $p(x,t)$ constant, the population will approach whichever point it is initially closer to. 

\begin{equation}
    \frac{dx}{dt} = \sigma x(1 - x)p(x,t) 
    \label{replicator}
\end{equation}


This formulation has been also used to model vaccination sentiment in a variety of scenarios \cite{oraby2014influence,bauch2004vaccination,bauch2005imitation,bauch2012evolutionary}. In this context, ``cooperation" refers to the strategy of getting a widely-available vaccine, and the cooperation payoff function is usually of the form in equation \ref{vacpayoff}. The population is assumed to have a constant payoff to avoid vaccination (in many cases, just due to inconvenience) and a payoff to vaccinate proportional to $I$, the prevalence of infection in the model. 

\begin{equation}
    p(x,t) = - c + \rho I
    \label{vacpayoff}
\end{equation}


A prisoners dilemma formulation and model based upon equation \ref{replicator} coupled to an application-specific model (in the above case, disease dynamics), can also be applied to human-environment models in ecology. In particular, it has been used to model conservation responses coupled to ecosystem dynamics in contexts such as forest-grassland mosaics \cite{innes2013impact,henderson2016alternative}, global climate \cite{bury2019charting}, coral reefs \cite{thampi2018socio}, argicultural land use \cite{gooding2018forest}. I will focus on the application of imitation dynamics to forest pest transport, and the use of NPIs in the context of the Covid-19 pandemic.


\section{Forest pests in eastern North America}

The term ``forest pests" covers a broad range of infectious agents that are reponsible for forest tree damage and mortality. Major invading forest pests in eastern North America include: the Asian longhorned beetle (\textit{Anoplophora glabripennis}), the butternut canker (\textit{Ophiognomonia clavigignenti-juglandacearum}), \textit{Lymantria dispar dispar}, dutch elm disease \textit{Ophiostoma ulmi}, and the aforementioned EAB. Together, these non-native pests kill 5.53 teragrams of carbon worth of trees each year, on an order of magnitude comparable to forest fires on the continent \cite{fei2019biomass}. Non-native insect invaders are usually introduced by accident. The majority of recently introduced species are a result of careless global trade, with new individuals arriving in lumber, live plants, or similar goods \cite{brockerhoff2017ecology}. Models for the spread of these insects are often inspired by models for infectious diseases in humans. Research in mathematical ecology generally uses the related lotka-volterra model for host-parasitoid dynamics, but SIR models can be a natural choice because they focus on the time-evolution of the host populations, which is often the more useful quantity \cite{edelstein2005mathematical}. In chapter \ref{ch2} we will expand on the model of Barlow et al \cite{barlow2014modelling}, which couples an SIR-style model of an invading species to human travel patterns, a common vector for forest pests \cite{buck2009hitchhiking,kolar2001progress,wilson2009something}.

\section{Mountain pine beetle (MPB) and fire-driven forest ecosystems of the Western Cordillera}

The coniferous forests of the western cordillera of North America are the subject of the model presented in chapter \ref{ch3}. The Canadian section of these forests are primarily composed of a mixture of \textit {Pinus} sp., namely lodgepole pine (\textit{Pinus Contorta}), but also ponderosa pine (\textit{Pinus ponderosa}) \cite{brown2010impact}. The fire regimes in these forests are generally characterized by frequent, low to mixed severity fires depending on elevation and climactic conditions \cite{agee1996fire,arno1980forest}. In these regions, there are also a few other dominant forest types: those dominated by Douglas Fir (\textit{Pseudotsuga menziesii}), and those dominated by subalpine spruce (\textit{Abies lasiocarpa}). These other forest types become dominant in areas which experience wetter or cooler climates, as they are less drought tolerant, and also less fire resistant \cite{JENKINS200816}. Therefore, the lodgepole pine forests are dependent on a frequent fire regime to maintain climax lodgepole forests. They are very rapidly growing when young, possess (usually) serotinous cones, and maintain massive seed banks in the soil to rapidly colonize the area after disturbances \cite{lotan1976cone,lotan1985role}.

Besides wildfire, MPB (\textit{Dendroctonus ponderosae}) is the most significant disturbance in these forest types. Endemic to this ecosystem, MPB most commonly attacks lodgepole pine in Canada \cite{safranyik2007mountain}, but it can attack and reproduce within all of the pine species in North America, and during outbreaks has been recorded to attack spruce and fir trees within its range \cite{gibson2009mountain}. MPB, and bark beetles more generally, exhibit highly cyclic lifestyles. For most of the year, they exist in the phloem of the tree first as eggs, then as larvae, until they are mature enough to emerge and fly to new hosts. The emergence of MPB occurs in late summer, although it is heavily dependent on the climate that year \cite{bentz2014mountain}. While their flight capability is limited, MPB can use air currents to colonize trees over 20km away from their place of birth \cite{shegelski2019morphological}. When individuals find a suitable host, they release phermones that attract other flying beetles and triggering a mass attack behaviour. This behaviour functions to overwhelm the defenses of the host tree. A successful attack results in the MPB laying their eggs in the phloem of the new host tree, and the cycle repeats. Older trees with thicker phloem are most susceptible to MPB attack, and they are generally the first to be colonized, with MPB attacking progressively less suitable hosts as population densities rise \cite{safranyik2007mountain}. Endemic periods of low MPB density give rise to outbreaks based on a variety of factors, such as density of good hosts, climate, and possibly wildfire damage \cite{safranyik2007mountain}. Although MPB has always exhibited outbreak cycles, in the past two decades, outbreak sizes have exceeded historically recorded levels probably due to increases in winter temperatures and higher densities of mature trees \cite{bentz2010climate,safranyik2007mountain}. Recently, jack pine \textit{Pinus banksiana} stands in northern Alberta, and the Northwest Territories, have been attacked by MPB as they expand their range north and eastward \cite{cudmore2010climate,nrcanmpb}. Understanding the holistic dynamics of these ecosystems, and the role that MPB takes within them, will be key to understanding the causes and effects of these unprecedented population levels. 

\section{Thesis Outline}

The remainder of this thesis is divided into three sections, each studying a coupled compartmental model specialized to a particular domain, followed by a synthesis and summary of the results from each chapter. In the first chapter, an age-structured impulsive differential equation model of Covid-19 is coupled to the aforementioned imitation dynamics for physical distancing. It is parameterized with case data from Ontario, Canada and population location data from Google. Two primary categories of vaccination strategy were considered in this model: vaccination of the most vulnerable populations (older age groups), or vaccination of the most transmitting populations (according to contact distribution estimates). We analyze how the timing, supply rate, and shutdown policies will affect the best vaccination policy through numerical simulation of the model.

The second chapter extends the forest pest and firewood transport model of Barlow et al. \cite{barlow2014modelling} to a large empirically derived network of human movement patterns between susceptible forest patches. Numerical analysis of this model is done to compare the effectiveness of three major policy categories in reducing the spread of invasive pests throughout forested areas in Eastern Canada. We consider direct interception of human-mediated transport of forest pests, changing behavioural incentives to transport firewood, and quarantine of the most central areas, and combinations thereof. These strategies are assessed with respect to total tree infections over periods of 5, 10, and 20 years.

The third chapter extends an age-structured, discrete time model of mountain pine beetle population \cite{duncan2015model} to include a simplified model of yearly burn sizes. The effect of changing fire disturbance regimes on the forest stand structure is explored through numerical simulations of the parameter space. Since MPB outbreak patterns seem to strongly depend on the density of mature trees, they are therefore sensitive to stand structure, in particular the creation of large-even aged stands created by severe forest fires. We discuss the dynamical regimes of this system, and argue that outbreak dynamics can be significantly influenced by heterogeneity in stand structure. The final chapter will summarize and contextualize these results, discuss their limitations, and outline opportunities for future work.