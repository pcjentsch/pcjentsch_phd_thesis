\chapter{Introduction}

I am writing this introduction sitting in my living room in Kitchener, Ontario, staring at the dead ash tree which overlooks the empty lots that border my home. Ash trees (\textit{Fraxinus} sp.) were a common street tree in the deciduous forests of the eastern Unites States and Canada, but in the past two decades, the emerald ash borer (\textit{Agrilus planipennis}) has spread further and further each year, killing about 99\% of the Ash trees in the regions they invade \cite{nrcaneab}. Once a prominent feature of deciduous woodlands in this area, \textit{Fraxinus} is now limited to rare pockets that have escaped the insect, and recently-germinated seedlings still too small to be infested. This narrative is not new. The american chestnut (\texit{Castanea dentata}) was once a major part of the eastern american landscape, an important source of lumber, and food. It was almost completely wiped out as the chestnut blight (\textit{Cryphonectria parasitica}) spread throughout eastern north america in the 19th and 20th centuries. Infectious agents in this way shape the landscapes we inhabit and the ecosytems we exist within. Of course, infectious agents are not limited to arboreal hosts: I have been in my living room staring at this dead ash tree for the past year, sheltering from the global Covid-19 pandemic. 

Pandemics, such as the various waves of the Black Death, the 1918 influenza pandemic, the HIV/AIDS pandemic, and the current Covid-19 pandemic, have irreversibly shaped our culture. Endemic infectious disease, was until very recently, a massive driving force in the formation of human societies everywhere. Only in the past century have some parts of the world been able to escape the spectre of endemic disease such as malaria, polio, influenza, and measles. John Snow is considered to be one of the first epidemiologists, for his study of London Cholera outbreaks \cite{snow1855mode, brauer2019mathematical}. However, during recent outbreaks of infectious agents, compartmental models are one of the main tools used to forecast outcomes and mitigation stragies \cite{brauer2008compartmental}. Due to the ongoing pandemic, the general public is probably more aware than ever before of these models.

Compartmental models for the spread of infectious diseases are usually considered to have been introduced to the field by public health researchers, such as Hamer, Kermack and McKendrick \cite{hamer1906epidemic, kermack1927contribution, brauer2019mathematical,edelstein2005mathematical}. This class of models divides the population into homogenous compartments, and describe the rate of movement between these compartments. The quintessential compartmental model is the SIR model (equations \ref{SIR}), so-called for it's division of a population into Susceptible (the $S(t)$ variable), Infected (the $I(t)$ variable), and Recovered (the $R(t)$ variable) compartments.

\begin{eqnarray}
    \frac{dS}{dt} = -\beta S I  \\
    \frac{dI}{dt} = \beta S I - \gamma I\\
    \frac{dR}{dt} = \gamma I\\
    \label{SIR}
\end{eqnarray}

The SIR model describes a population undergoing an infection that confers complete immunity after one has been infected, under a great number of simplying assumptions: each compartment is totally homogenous, the likelihood of infection is proportional to the number of individuals in each compartment, and the probability that an individual recovers is constant per unit time. From the initial work on these models, the invention of computers has allowed researchers to get useful results from even the most complex elaborations on this theme. The assumption of homogeneity is a major drawback of the SIR model, so a common extension to this model is to add more structure, usually through furthur subdivision of the compartments. 



\section{Sars-Cov-2}

Compartmental models have been invaluable in modeling the spread of Sars-CoV-2 \cite{thompson2020epidemiological}. It is difficult to overstate the damage and loss of life that the ongoing COVID-19 pandemic has caused or exacerbated in the past year \cite{miller2020disease,who2021impact}. The first human case of Sars-CoV-2 occurred in late 2019 in central China, almost certainly transmitted from an animal host, very likely a bat \cite{andersen2020proximal,rasmussen2021origins,zhu2020novel}. The virus soon spread throughout the world, and was declared a pandemic by the World Health Organization (WHO), on March 11th, 2020 \cite{who2020announces}. In the intervening months, people around the world have endured various levels of non-pharmeceutical interventions (NPIs), from comprehensive quarantine procedures (in e.g. New Zealand, South Korea, Singapore, Vietnam) to almost nothing at all (United States, Sweden). In the outcomes of the aforementioned countries, empirical research, and modelling studies, NPIs have been shown to be an effective method for the control of Covid-19 \cite{anderson2020estimating,flaxman2020estimating,ferguson2020report,demirguc2020sooner}. 
% It is a consequence of the complete failure of our social structures, leadership, and economic system to centre human well-being.  



\section{Imitation dynamics}


The demonstrated effectiveness of NPIs in some countries implies that the immense morbidity and mortality over the past year is not simply a natural disaster, but a humanitarian one. Since our survival depends on our ability to construct a world in which people are incentivized to centre the well-being of others, any attempt to model human outcomes of the pandemic should also attempt to model the incentive strucutures we operate within. A very simple model for this is the game theoretical one \cite{andrews2015disease,jentsch2018spatial}. In this framework, we assume that NPI usage is a prisoners dilemma in that everyone either cooperates or defects with the practice of using NPIs, and the decision to defect to cooperate is based on a combination of the percieved payoff to do so, and the influence of the rest of the population. Table \ref{prisonersdilemma} shows the payoff matrix of this 2-player game. Of course in real life, we are all playing this game, all the time, with everyone. 
\begin{figure}
    \begin{tabular}{ |c|c| c| } \hline
        \diagbox[width = 7em, height = 2em]{P1}{P2} &use NPI& don't use NPI   \\ \hline
        use NPI & \diagbox[width = 13em, height = 8em]{low risk,\\ NPIs unpleasant}{low risk,\\ NPIs unpleasant} &  \diagbox[width = 13em, height = 8em]{med risk,\\ NPIs unpleasant} {med risk}\\ \hline 
        don't use NPI & \diagbox[width = 13em, height = 8em]{med risk}{med risk,\\ NPIs unpleasant} &  \diagbox[width = 13em, height = 8em]{high risk}{high risk}   \\ \hline
    \end{tabular}
    \caption{NPI adoption as a two-player game (between P1 and P2)}
    \label{prisonersdilemma}
\end{figure}


The simplest way to approximate the time evolution of a such a game is with the one-dimensional replicator equation, which approximates these dynamics in terms of the population average \cite{hofbauer1998evolutionary}. Specifically, we introduce a variable $x(t)$ which represents the fraction of people adopting a strategy, then the replicator equation \ref{replicator} gives the time-evolution of $x(t)$ in terms of the payoff for cooperating over defecting, $p(x,t)$. We see immediately that this equation, disregarding $p(x,t) = 0$, has two steady states: $x = 1$ and $x = 0$. Given $p(x,t)$ constant, the population will approach whichever point it is initially closer to. 

\begin{equation}
    \frac{dx}{dt} = \sigma x(1 - x)p(x,t) 
    \label{replicator}
\end{equation}


This formulation has been also used to model vaccination sentiment in a variety of scenarios \cite{oraby2014influence,bauch2004vaccination,bauch2005imitation,bauch2012evolutionary}. In this context, "cooperation" refers to the strategy of getting a widely-available vaccine, and the cooperation payoff function is usually of the form in equation \ref{vacpayoff}. The population is assumed to have a constant payoff to avoid vaccination (in many cases, just due to inconvenience), and a payoff to vaccinate proportional to $I$, the prevalence of infection in the model. 

\begin{equation}
    p(x,t) = - c + \rho I
    \label{vacpayoff}
\end{equation}


A prisoners dilemma formulation and model based upon equation \ref{replicator} coupled to an application-specific model (in the above case, disease dynamics), can also be applied to human-environment models in ecology. In particular, it has been used to model conservation responses coupled to ecosystem dynamics in contexts such as forest-grassland mosaics \cite{innes2013impact,henderson2016alternative}, global climate \cite{bury2019charting}, coral reefs \cite{thampi2018socio}, argicultural land use \cite{gooding2018forest}. I will focus on the application of imitation dynamics to forest pest transport \cite{barlow2014modelling}, and the use of NPIs in the context of the Covid-19 pandemic. 


\section{Forest pests and firewood transport}

Compartmental models are not limited to infections in humans, they can also be applied to ecological systems. While research in mathematical ecology generally uses the related lotka-volterra model for host-parasitoid dynamics, SIR models can be a natural choice because they centre the time-evolution of the host populations, which is often the more useful quantity \cite{edelstein2005mathematical}. 


\section{Mountain pine beetle}


% \addcontentsline{toc}{chapter}{Introduction} \markboth{INTRODUCTION}{} \lipsum[1-5]

% The growth of an organism within a population over time is a common subject of research in ecology and the natural sciences. Some of the first dynamic models of this process were created by Lokta and Volterra in the early 20th century \cite{lotka1920analytical,volterra1928variations,lotka1925elements}. The lotka-volterra model (Equations \ref{lotkavolterra}) describes the time-evolution of two populations, a predator and a prey, under many simplifying assumptions. Namely, this model assuming that prey grows exponentially in the abscence of a predator, and that populations are completely homogenous in space, the growth rate of predators depends entirely on the population of prey, and that population sizes are the only factor governing the rate of predation.


% \begin{eqnarray}
%     \frac{dx}{dt} = \alpha x - \beta x y \\
%     \frac{dy}{dt} = \gamma x y - \delta y\\
%     \label{lotkavolterra}
% \end{eqnarray}

% The spread of a 