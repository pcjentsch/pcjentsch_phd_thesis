%======================================================================
% University of Waterloo Thesis Template for LaTeX 
% Last Updated November, 2020 
% by Stephen Carr, IST Client Services, 
% University of Waterloo, 200 University Ave. W., Waterloo, Ontario, Canada
% FOR ASSISTANCE, please send mail to request@uwaterloo.ca

% DISCLAIMER
% To the best of our knowledge, this template satisfies the current uWaterloo thesis requirements.
% However, it is your responsibility to assure that you have met all requirements of the University and your particular department.

% Many thanks for the feedback from many graduates who assisted the development of this template.
% Also note that there are explanatory comments and tips throughout this template.
%======================================================================
% Some important notes on using this template and making it your own...

% The University of Waterloo has required electronic thesis submission since October 2006. 
% See the uWaterloo thesis regulations at
% https://uwaterloo.ca/graduate-studies/thesis.
% This thesis template is geared towards generating a PDF version optimized for viewing on an electronic display, including hyperlinks within the PDF.

% DON'T FORGET TO ADD YOUR OWN NAME AND TITLE in the "hyperref" package configuration below. 
% THIS INFORMATION GETS EMBEDDED IN THE PDF FINAL PDF DOCUMENT.
% You can view the information if you view properties of the PDF document.

% Many faculties/departments also require one or more printed copies. 
% This template attempts to satisfy both types of output. 
% See additional notes below.
% It is based on the standard "book" document class which provides all necessary sectioning structures and allows multi-part theses.

% If you are using this template in Overleaf (cloud-based collaboration service), then it is automatically processed and previewed for you as you edit.

% For people who prefer to install their own LaTeX distributions on their own computers, and process the source files manually, the following notes provide the sequence of tasks:
 
% E.g. to process a thesis called "mythesis.tex" based on this template, run:

% pdflatex mythesis	-- first pass of the pdflatex processor
% bibtex mythesis	-- generates bibliography from .bib data file(s)
% makeindex         -- should be run only if an index is used 
% pdflatex mythesis	-- fixes numbering in cross-references, bibliographic references, glossaries, index, etc.
% pdflatex mythesis	-- it takes a couple of passes to completely process all cross-references

% If you use the recommended LaTeX editor, Texmaker, you would open the mythesis.tex file, then click the PDFLaTeX button. Then run BibTeX (under the Tools menu).
% Then click the PDFLaTeX button two more times. 
% If you have an index as well,you'll need to run MakeIndex from the Tools menu as well, before running pdflatex
% the last two times.

% N.B. The "pdftex" program allows graphics in the following formats to be included with the "\includegraphics" command: PNG, PDF, JPEG, TIFF
% Tip: Generate your figures and photos in the size you want them to appear in your thesis, rather than scaling them with \includegraphics options.
% Tip: Any drawings you do should be in scalable vector graphic formats: SVG, PNG, WMF, EPS and then converted to PNG or PDF, so they are scalable in the final PDF as well.
% Tip: Photographs should be cropped and compressed so as not to be too large.

% To create a PDF output that is optimized for double-sided printing: 
% 1) comment-out the \documentclass statement in the preamble below, and un-comment the second \documentclass line.
% 2) change the value assigned below to the boolean variable "PrintVersion" from " false" to "true".

%======================================================================
%   D O C U M E N T   P R E A M B L E
% Specify the document class, default style attributes, and page dimensions, etc.
% For hyperlinked PDF, suitable for viewing on a computer, use this:
\documentclass[letterpaper,12pt,titlepage,oneside,final]{book}
 
% For PDF, suitable for double-sided printing, change the PrintVersion variable below to "true" and use this \documentclass line instead of the one above:
%\documentclass[letterpaper,12pt,titlepage,openright,twoside,final]{book}

% Some LaTeX commands I define for my own nomenclature.
% If you have to, it's easier to make changes to nomenclature once here than in a million places throughout your thesis!
\newcommand{\package}[1]{\textbf{#1}} % package names in bold text
\newcommand{\cmmd}[1]{\textbackslash\texttt{#1}} % command name in tt font 
\newcommand{\href}[1]{#1} % does nothing, but defines the command so the print-optimized version will ignore \href tags (redefined by hyperref pkg).
%\newcommand{\texorpdfstring}[2]{#1} % does nothing, but defines the command
% Anything defined here may be redefined by packages added below...

% This package allows if-then-else control structures.
\usepackage{ifthen}
\newboolean{PrintVersion}
\setboolean{PrintVersion}{false}
% CHANGE THIS VALUE TO "true" as necessary, to improve printed results for hard copies by overriding some options of the hyperref package, called below.

%\usepackage{nomencl} % For a nomenclature (optional; available from ctan.org)
\usepackage{amsmath,amssymb,amstext} % Lots of math symbols and environments
\usepackage[pdftex]{graphicx} % For including graphics N.B. pdftex graphics driver 

% Hyperlinks make it very easy to navigate an electronic document.
% In addition, this is where you should specify the thesis title and author as they appear in the properties of the PDF document.
% Use the "hyperref" package 
% N.B. HYPERREF MUST BE THE LAST PACKAGE LOADED; ADD ADDITIONAL PKGS ABOVE
\usepackage[pdftex,pagebackref=false]{hyperref} % with basic options
%\usepackage[pdftex,pagebackref=true]{hyperref}
		% N.B. pagebackref=true provides links back from the References to the body text. This can cause trouble for printing.
\hypersetup{
    plainpages=false,       % needed if Roman numbers in frontpages
    unicode=false,          % non-Latin characters in Acrobat’s bookmarks
    pdftoolbar=true,        % show Acrobat’s toolbar?
    pdfmenubar=true,        % show Acrobat’s menu?
    pdffitwindow=false,     % window fit to page when opened
    pdfstartview={FitH},    % fits the width of the page to the window
%    pdftitle={uWaterloo\ LaTeX\ Thesis\ Template},    % title: CHANGE THIS TEXT!
%    pdfauthor={Author},    % author: CHANGE THIS TEXT! and uncomment this line
%    pdfsubject={Subject},  % subject: CHANGE THIS TEXT! and uncomment this line
%    pdfkeywords={keyword1} {key2} {key3}, % list of keywords, and uncomment this line if desired
    pdfnewwindow=true,      % links in new window
    colorlinks=true,        % false: boxed links; true: colored links
    linkcolor=blue,         % color of internal links
    citecolor=green,        % color of links to bibliography
    filecolor=magenta,      % color of file links
    urlcolor=cyan           % color of external links
}
\ifthenelse{\boolean{PrintVersion}}{   % for improved print quality, change some hyperref options
\hypersetup{	% override some previously defined hyperref options
%    colorlinks,%
    citecolor=black,%
    filecolor=black,%
    linkcolor=black,%
    urlcolor=black}
}{} % end of ifthenelse (no else)

\usepackage[automake,toc,abbreviations]{glossaries-extra} % Exception to the rule of hyperref being the last add-on package
% If glossaries-extra is not in your LaTeX distribution, get it from CTAN (http://ctan.org/pkg/glossaries-extra), 
% although it's supposed to be in both the TeX Live and MikTeX distributions. There are also documentation and 
% installation instructions there.

% Setting up the page margins...
% uWaterloo thesis requirements specify a minimum of 1 inch (72pt) margin at the
% top, bottom, and outside page edges and a 1.125 in. (81pt) gutter margin (on binding side). 
% While this is not an issue for electronic viewing, a PDF may be printed, and so we have the same page layout for both printed and electronic versions, we leave the gutter margin in.
% Set margins to minimum permitted by uWaterloo thesis regulations:
\setlength{\marginparwidth}{0pt} % width of margin notes
% N.B. If margin notes are used, you must adjust \textwidth, \marginparwidth
% and \marginparsep so that the space left between the margin notes and page
% edge is less than 15 mm (0.6 in.)
\setlength{\marginparsep}{0pt} % width of space between body text and margin notes
\setlength{\evensidemargin}{0.125in} % Adds 1/8 in. to binding side of all 
% even-numbered pages when the "twoside" printing option is selected
\setlength{\oddsidemargin}{0.125in} % Adds 1/8 in. to the left of all pages when "oneside" printing is selected, and to the left of all odd-numbered pages when "twoside" printing is selected
\setlength{\textwidth}{6.375in} % assuming US letter paper (8.5 in. x 11 in.) and side margins as above
\raggedbottom

% The following statement specifies the amount of space between paragraphs. Other reasonable specifications are \bigskipamount and \smallskipamount.
\setlength{\parskip}{\medskipamount}

% The following statement controls the line spacing.  
% The default spacing corresponds to good typographic conventions and only slight changes (e.g., perhaps "1.2"), if any, should be made.
\renewcommand{\baselinestretch}{1} % this is the default line space setting

% By default, each chapter will start on a recto (right-hand side) page.
% We also force each section of the front pages to start on a recto page by inserting \cleardoublepage commands.
% In many cases, this will require that the verso (left-hand) page be blank, and while it should be counted, a page number should not be printed.
% The following statements ensure a page number is not printed on an otherwise blank verso page.
\let\origdoublepage\cleardoublepage
\newcommand{\clearemptydoublepage}{%
  \clearpage{\pagestyle{empty}\origdoublepage}}
\let\cleardoublepage\clearemptydoublepage

% Define Glossary terms (This is properly done here, in the preamble and could also be \input{} from a separate file...)
% Main glossary entries -- definitions of relevant terminology
\newglossaryentry{computer}
{
name=computer,
description={A programmable machine that receives input data,
               stores and manipulates the data, and provides
               formatted output}
}

% Nomenclature glossary entries -- New definitions, or unusual terminology
\newglossary*{nomenclature}{Nomenclature}
\newglossaryentry{dingledorf}
{
type=nomenclature,
name=dingledorf,
description={A person of supposed average intelligence who makes incredibly brainless misjudgments}
}

% List of Abbreviations (abbreviations type is built in to the glossaries-extra package)
\newabbreviation{aaaaz}{AAAAZ}{American Association of Amateur Astronomers and Zoologists}

% List of Symbols
\newglossary*{symbols}{List of Symbols}
\newglossaryentry{rvec}
{
name={$\mathbf{v}$},
sort={label},
type=symbols,
description={Random vector: a location in n-dimensional Cartesian space, where each dimensional component is determined by a random process}
}
\makeglossaries

%======================================================================
%   L O G I C A L    D O C U M E N T
% The logical document contains the main content of your thesis.
% Being a large document, it is a good idea to divide your thesis into several files, each one containing one chapter or other significant chunk of content, so you can easily shuffle things around later if desired.
%======================================================================
\begin{document}
%------------------------------------
----------------------------------
% FRONT MATERIAL
% title page,declaration, borrowers' page, abstract, acknowledgements,
% dedication, table of contents, list of tables, list of figures, nomenclature, etc.
%----------------------------------------------------------------------
% T I T L E   P A G E
% -------------------
% Last updated October 23, 2020, by Stephen Carr, IST-Client Services
% The title page is counted as page `i' but we need to suppress the
% page number. Also, we don't want any headers or footers.
\pagestyle{empty}
\pagenumbering{roman}

% The contents of the title page are specified in the "titlepage"
% environment.
\begin{titlepage}
        \begin{center}
        \vspace*{1.0cm}

        \Huge
        {\bf Coupled models of structured contagious processes}

        \vspace*{1.0cm}

        \normalsize
        by \\

        \vspace*{1.0cm}

        \Large
        Peter C. Jentsch \\

        \vspace*{3.0cm}

        \normalsize
        A thesis \\
        presented to the University of Waterloo \\ 
        in fulfillment of the \\
        thesis requirement for the degree of \\
        Doctor of Philosophy \\
        in \\
        Applied Mathematics \\

        \vspace*{2.0cm}

        Waterloo, Ontario, Canada, 2021 \\

        \vspace*{1.0cm}

        \copyright\ Peter C. Jentsch \\
        \end{center}
\end{titlepage}

% The rest of the front pages should contain no headers and be numbered using Roman numerals starting with `ii'
\pagestyle{plain}
\setcounter{page}{2}

\cleardoublepage % Ends the current page and causes all figures and tables that have so far appeared in the input to be printed.
% In a two-sided printing style, it also makes the next page a right-hand (odd-numbered) page, producing a blank page if necessary.

 
% E X A M I N I N G   C O M M I T T E E (Required for Ph.D. theses only)
% Remove or comment out the lines below to remove this page
\begin{center}\textbf{Examining Committee Membership}\end{center}
  \noindent
The following served on the Examining Committee for this thesis. The decision of the Examining Committee is by majority vote.
  \bigskip
  
  \noindent
\begin{tabbing}
Internal-External Member: \=  \kill % using longest text to define tab length
External Examiner: \>  Bruce Bruce \\ 
\> Professor, Dept. of Philosophy of Zoology, University of Wallamaloo \\
\end{tabbing} 
  \bigskip
  
  \noindent
\begin{tabbing}
Internal-External Member: \=  \kill % using longest text to define tab length
Supervisors: \> Chris T. Bauch \\
\> Professor, Department of Applied Mathematics, University of Waterloo \\
\> Madhur Anand \\
\> Professor, School of Environmental Sciences, University of Guelph \\
\end{tabbing}
  \bigskip
  
  \noindent
  \begin{tabbing}
Internal-External Member: \=  \kill % using longest text to define tab length
Internal Members: \> Sue Ann Campbell \\
\> Professor, Department of Applied Mathematics, University of Waterloo \\ 
\> Zoran Miskovic  \\
\> Professor, Department of Applied Mathematics, University of Waterloo \\
\end{tabbing}
  \bigskip
  
  \noindent
\begin{tabbing}
Internal-External Member: \=  \kill % using longest text to define tab length
Internal-External Member: \> Meta Meta \\
\> Professor, Dept. of Philosophy, University of Waterloo \\
\end{tabbing}
  \bigskip
  
  \noindent
\begin{tabbing}
Internal-External Member: \=  \kill % using longest text to define tab length
Other Member(s): \> Leeping Fang \\
\> Professor, Dept. of Fine Art, University of Waterloo \\
\end{tabbing}

\cleardoublepage

% D E C L A R A T I O N   P A G E
% -------------------------------
  % The following is a sample Delaration Page as provided by the GSO
  % December 13th, 2006.  It is designed for an electronic thesis.
 \begin{center}\textbf{Author's Declaration}\end{center}
  
 \noindent
This thesis has entirely been authored or co-authored by me. This is a true copy of the thesis, including any required final revisions, as accepted by my examiners.

  \bigskip
  
  \noindent
I understand that my thesis may be made electronically available to the public.

\cleardoublepage

% Contributions
% -------------------------------
  % The following is a sample Delaration Page as provided by the GSO
  % December 13th, 2006.  It is designed for an electronic thesis.
  \begin{center}\textbf{Statement of Contributions}\end{center}
  
  \begin{itemize}
    
   \item Chapter \ref{ch1}: MA and CTB conceptualized the study. All authors designed the model. PCJ developed and analysed the model and generated figures. PCJ and CTB wrote the first draft of the manuscript and accessed and verified the data. All authors revised the manuscript critically for important intellectual content, and read and approved the final version of the manuscript. All authors had full access to all the data in the study, and the corresponding author had final responsibility for the decision to submit for publication. The contents of this chapter are based on the corresponding article published in \textit{Lancet Infectious Diseases} \cite{jentsch2021prioritising}.
   
   \item Chapter \ref{ch2}: Conceptualization by DY, MA, data curation by PCJ, DY, formal analysis by PCJ, funding acquisition by CTB, DY, MA, investigation by PCJ, CTB, DY, MA. Methodology by PCJ, CTB, DY, MA. Project administration by PCJ, CTB, DY, MA. Resources by CTB, DY, MA. Software written by PCJ. Supervision by CTB, DY, MA, validation by PCJ, CTB, DY, MA, visualization by PCJ, CTB. Original draft written by PCJ. Review & editing by PCJ, CTB, DY, MA. The contents of this chapter are based on the corresponding article published in \textit{PLoS One} \cite{jentsch2020go}.

   
   \item Chapter \ref{ch3}: The work in this chapter is based upon a manuscript under review at the \textit{Journal of Theoretical Ecology}. All authors conceived ideas for the study. PCJ designed and coded the model, performed
   analyses, created figures, and drafted the manuscript. All authors revised the manuscript.
   

  \end{itemize}
 \cleardoublepage
 
% A B S T R A C T
% ---------------

\begin{center}\textbf{Abstract}\end{center}

% In 2021, the biosphere of the earth has been almost entirely eradicated and replaced by a thin facsimile. Many of the ecosystems that once spanned the globe have been reduced to skeletons, supported by the scaffolding of dwindling state environmental agency budgets. 

Models of infectious processes are a common feature in the landscape of applied mathematics. Is is very rare that these processes are isolated from other significant dynamics in nature, and therefore we can incorporate some of the complexity inherent in real systems by coupling infections to major features of the ecosystems they inhabit. Infectious processes can take many forms, but in this thesis we consider three: the COVID-19 pandemic, the invasion of eastern North American forests by wood-borne pests, and the outbreak cycles of an endemic forest pest. The first chapter covers a model of Sars-CoV-2 in a structured population, coupled with a replicator equation representing sentiment towards the use of non-pharmaceutical interventions. We use this model to compare the efficacy of vulnerable-first and transmission-preventing age structured vaccination strategies. The buildup of natural immunity in a population combined with a low vaccination supply, is shown to cause a transmission-preventing vaccination strategy to be more effective. The second chapter considers a model of forest pest transport over an empirically-derived network of forest patches in eastern Canada. Since these pests can frequently be spread long distances by wood transport, we couple this model to the sentiment of local populations towards avoiding firewood transport from outside their area. Three possible countermeasures to the spread of the invasive pest are compared: social incentives, direct interception of infested firewood, and quarantine of patches. The level of effort needed to significantly reduce forest damage with any of these methods is substantial and unlikely to be implemented. The final chapter extends a model of mountain pine beetle (MPB) in western north american pine forests to incorporate tree mortality due to wildfire. We find that wildfire acts as a disturbance that increases the heterogeneity in age structure, and therefore is able to increase the resilience of the forest to outbreaks of MPB. A targeted thinning procedure aimed specifically at increasing heterogeneity in the forest age structure is proposed and shown to be highly effective at reducing the severity of outbreak. The effectiveness of targeted thinning in the manner described further emphasizes the importance of heterogeneity in forest stand structure.



\cleardoublepage

% A C K N O W L E D G E M E N T S
% -------------------------------

\begin{center}\textbf{Acknowledgements}\end{center}

I would like to thank the following people for making this thesis possible.

\begin{itemize}
\item My supervisors, Dr. Chris Bauch and Dr. Madhur Anand, for their endless support, understanding, and insight throughout my graduate and undergraduate career.  
  
\item Dr. Sue Ann Campbell and Zoran Miskovic for being on my PhD committee.

\item Dr. Denys Yemshanov for his patience and mentorship on things forest-related. 

\item Dr. Mark Penney for the great conversations, collaboration, and opportunity to write an agent-based model in the last six months of my PhD.

\item Dr. Chrystopher Nehaniv for the opportunities to think and present about topics almost completely different from those in this thesis. 

\item My friends in the Bauch Lab, the applied mathematics department, and outside the university for the time spent not working, and many victorious bar trivia nights.
 
\item Samantha Landry and Emylee Todd for reading and editing the early drafts of this thesis.

\item My mom, dad, and siblings, for their support, encouragement, and not asking "so when are you gonna graduate" too often.  
\end{itemize}

 
\cleardoublepage

% D E D I C A T I O N
% -------------------

\begin{center}\textbf{Dedication}\end{center}

For Emmy, Max, Sam, and the friends that have helped me get through the past 5 years.


\cleardoublepage

% T A B L E   O F   C O N T E N T S
% ---------------------------------
\renewcommand\contentsname{Table of Contents}
\tableofcontents
\cleardoublepage
\phantomsection    % allows hyperref to link to the correct page

% L I S T   O F   F I G U R E S
% -----------------------------
\addcontentsline{toc}{chapter}{List of Figures}
\listoffigures
\cleardoublepage
\phantomsection		% allows hyperref to link to the correct page

% L I S T   O F   T A B L E S
% ---------------------------
\addcontentsline{toc}{chapter}{List of Tables}
\listoftables
\cleardoublepage
\phantomsection		% allows hyperref to link to the correct page

% Change page numbering back to Arabic numerals
\pagenumbering{arabic}

 

%----------------------------------------------------------------------
% MAIN BODY
% We suggest using a separate file for each chapter of your thesis.
% Start each chapter file with the \chapter command.
% Only use \documentclass or \begin{document} and \end{document} commands in this master document.
% Tip: Putting each sentence on a new line is a way to simplify later editing.
%----------------------------------------------------------------------
%======================================================================
\chapter{Introduction}
%======================================================================
In the beginning, there was $\pi$:

\begin{equation}
   e^{\pi i} + 1 = 0  \label{eqn_pi}
\end{equation}
A \gls{computer} could compute $\pi$ all day long. In fact, subsets of digits of $\pi$'s decimal approximation would make a good source for psuedo-random vectors, \gls{rvec} . 

%----------------------------------------------------------------------
\section{State of the Art}
%----------------------------------------------------------------------

See equation \ref{eqn_pi} on page \pageref{eqn_pi}.\footnote{A famous equation.}

\section{Some Meaningless Stuff}

The credo of the \gls{aaaaz} was, for several years, several paragraphs of gibberish, until the \gls{dingledorf} responsible for the \gls{aaaaz} Web site realized his mistake:

"Velit dolor illum facilisis zzril ipsum, augue odio, accumsan ea augue molestie lobortis zzril laoreet ex ad, adipiscing nulla. Veniam dolore, vel te in dolor te, feugait dolore ex vel erat duis nostrud diam commodo ad eu in consequat esse in ut wisi. Consectetuer dolore feugiat wisi eum dignissim tincidunt vel, nostrud, at vulputate eum euismod, diam minim eros consequat lorem aliquam et ad. Feugait illum sit suscipit ut, tation in dolore euismod et iusto nulla amet wisi odio quis nisl feugiat adipiscing luptatum minim nisl, quis, erat, dolore. Elit quis sit dolor veniam blandit ullamcorper ex, vero nonummy, duis exerci delenit ullamcorper at feugiat ullamcorper, ullamcorper elit vulputate iusto esse luptatum duis autem. Nulla nulla qui, te praesent et at nisl ut in consequat blandit vel augue ut.

Illum suscipit delenit commodo augue exerci magna veniam hendrerit dignissim duis ut feugait amet dolor dolor suscipit iriure veniam. Vel quis enim vulputate nulla facilisis volutpat vel in, suscipit facilisis dolore ut veniam, duis facilisi wisi nulla aliquip vero praesent nibh molestie consectetuer nulla. Wisi nibh exerci hendrerit consequat, nostrud lobortis ut praesent dignissim tincidunt enim eum accumsan. Lorem, nonummy duis iriure autem feugait praesent, duis, accumsan tation enim facilisi qui te dolore magna velit, iusto esse eu, zzril. Feugiat enim zzril, te vel illum, lobortis ut tation, elit luptatum ipsum, aliquam dolor sed. Ex consectetuer aliquip in, tation delenit dignissim accumsan consequat, vero, et ad eu velit ut duis ea ea odio.

Vero qui, te praesent et at nisl ut in consequat blandit vel augue ut dolor illum facilisis zzril ipsum. Exerci odio, accumsan ea augue molestie lobortis zzril laoreet ex ad, adipiscing nulla, et dolore, vel te in dolor te, feugait dolore ex vel erat duis. Ut diam commodo ad eu in consequat esse in ut wisi aliquip dolore feugiat wisi eum dignissim tincidunt vel, nostrud. Ut vulputate eum euismod, diam minim eros consequat lorem aliquam et ad luptatum illum sit suscipit ut, tation in dolore euismod et iusto nulla. Iusto wisi odio quis nisl feugiat adipiscing luptatum minim. Illum, quis, erat, dolore qui quis sit dolor veniam blandit ullamcorper ex, vero nonummy, duis exerci delenit ullamcorper at feugiat. Et, ullamcorper elit vulputate iusto esse luptatum duis autem esse nulla qui.

Praesent dolore et, delenit, laoreet dolore sed eros hendrerit consequat lobortis. Dolor nulla suscipit delenit commodo augue exerci magna veniam hendrerit dignissim duis ut feugait amet. Ad dolor suscipit iriure veniam blandit quis enim vulputate nulla facilisis volutpat vel in. Erat facilisis dolore ut veniam, duis facilisi wisi nulla aliquip vero praesent nibh molestie consectetuer nulla, iriure nibh exerci hendrerit. Vel, nostrud lobortis ut praesent dignissim tincidunt enim eum accumsan ea, nonummy duis. Ad autem feugait praesent, duis, accumsan tation enim facilisi qui te dolore magna velit, iusto esse eu, zzril vel enim zzril, te. Nisl illum, lobortis ut tation, elit luptatum ipsum, aliquam dolor sed minim consectetuer aliquip.

Tation exerci delenit ullamcorper at feugiat ullamcorper, ullamcorper elit vulputate iusto esse luptatum duis autem esse nulla qui. Volutpat praesent et at nisl ut in consequat blandit vel augue ut dolor illum facilisis zzril ipsum, augue odio, accumsan ea augue molestie lobortis zzril laoreet. Ex duis, te velit illum odio, nisl qui consequat aliquip qui blandit hendrerit. Ea dolor nonummy ullamcorper nulla lorem tation laoreet in ea, ullamcorper vel consequat zzril delenit quis dignissim, vulputate tincidunt ut."
%======================================================================
\chapter{Observations}
%======================================================================

This would be a good place for some figures and tables.

Some notes on figures and photographs\ldots

\begin{itemize}
\item A well-prepared PDF should be 
  \begin{enumerate}
    \item Of reasonable size, {\it i.e.} photos cropped and compressed.
    \item Scalable, to allow enlargment of text and drawings. 
  \end{enumerate} 
\item Photos must be bit maps, and so are not scaleable by definition. TIFF and
BMP are uncompressed formats, while JPEG is compressed. Most photos can be
compressed without losing their illustrative value.
\item Drawings that you make should be scalable vector graphics, \emph{not} 
bit maps. Some scalable vector file formats are: EPS, SVG, PNG, WMF. These can
all be converted into PNG or PDF, that pdflatex recognizes. Your drawing 
package can probably export to one of these formats directly. Otherwise, a 
common procedure is to print-to-file through a Postscript printer driver to 
create a PS file, then convert that to EPS (encapsulated PS, which has a 
bounding box to describe its exact size rather than a whole page). 
Programs such as GSView (a Ghostscript GUI) can create both EPS and PDF from PS files.
Appendix~\ref{AppendixA} shows how to generate properly sized Matlab plots and save them as PDF.
\item It's important to crop your photos and draw your figures to the size that
you want to appear in your thesis. Scaling photos with the 
includegraphics command will cause loss of resolution. And scaling down 
drawings may cause any text annotations to become too small.
\end{itemize}

For more information on \LaTeX\, see these  \href{https://uwaterloo.ca/information-systems-technology/services/electronic-thesis-preparation-and-submission-support/ethesis-guide/creating-pdf-version-your-thesis/creating-pdf-files-using-latex/latex-ethesis-and-large-documents}{course notes}. 
\footnote{
Note that while it is possible to include hyperlinks to external documents,
it is not wise to do so, since anything you can't control may change over time. 
It \emph{would} be appropriate and necessary to provide external links to 
additional resources that you provide for a multimedia ``enhanced'' thesis. 
But also note that if the \package{hyperref} package is not included, 
as for the print-optimized option in this thesis template, any \cmmd{href} 
commands in your logical document are no longer defined.
A work-around employed by this thesis template is to define a dummy \cmmd{href} 
command (which does nothing) in the preamble of the document, 
before the \package{hyperref} package is included. 
The dummy definition is then redifined by the
\package{hyperref} package when it is included.
}

The classic book by Leslie Lamport \cite{lamport.book}, author of \LaTeX , is worth a look too, and the many available add-on packages are described by 
Goossens \textit{et al} \cite{goossens.book}.

%======================================================================
\chapter{Introduction}
%======================================================================
In the beginning, there was $\pi$:

\begin{equation}
   e^{\pi i} + 1 = 0  \label{eqn_pi}
\end{equation}
A \gls{computer} could compute $\pi$ all day long. In fact, subsets of digits of $\pi$'s decimal approximation would make a good source for psuedo-random vectors, \gls{rvec} . 

%----------------------------------------------------------------------
\section{State of the Art}
%----------------------------------------------------------------------

See equation \ref{eqn_pi} on page \pageref{eqn_pi}.\footnote{A famous equation.}

\section{Some Meaningless Stuff}

The credo of the \gls{aaaaz} was, for several years, several paragraphs of gibberish, until the \gls{dingledorf} responsible for the \gls{aaaaz} Web site realized his mistake:

"Velit dolor illum facilisis zzril ipsum, augue odio, accumsan ea augue molestie lobortis zzril laoreet ex ad, adipiscing nulla. Veniam dolore, vel te in dolor te, feugait dolore ex vel erat duis nostrud diam commodo ad eu in consequat esse in ut wisi. Consectetuer dolore feugiat wisi eum dignissim tincidunt vel, nostrud, at vulputate eum euismod, diam minim eros consequat lorem aliquam et ad. Feugait illum sit suscipit ut, tation in dolore euismod et iusto nulla amet wisi odio quis nisl feugiat adipiscing luptatum minim nisl, quis, erat, dolore. Elit quis sit dolor veniam blandit ullamcorper ex, vero nonummy, duis exerci delenit ullamcorper at feugiat ullamcorper, ullamcorper elit vulputate iusto esse luptatum duis autem. Nulla nulla qui, te praesent et at nisl ut in consequat blandit vel augue ut.

Illum suscipit delenit commodo augue exerci magna veniam hendrerit dignissim duis ut feugait amet dolor dolor suscipit iriure veniam. Vel quis enim vulputate nulla facilisis volutpat vel in, suscipit facilisis dolore ut veniam, duis facilisi wisi nulla aliquip vero praesent nibh molestie consectetuer nulla. Wisi nibh exerci hendrerit consequat, nostrud lobortis ut praesent dignissim tincidunt enim eum accumsan. Lorem, nonummy duis iriure autem feugait praesent, duis, accumsan tation enim facilisi qui te dolore magna velit, iusto esse eu, zzril. Feugiat enim zzril, te vel illum, lobortis ut tation, elit luptatum ipsum, aliquam dolor sed. Ex consectetuer aliquip in, tation delenit dignissim accumsan consequat, vero, et ad eu velit ut duis ea ea odio.

Vero qui, te praesent et at nisl ut in consequat blandit vel augue ut dolor illum facilisis zzril ipsum. Exerci odio, accumsan ea augue molestie lobortis zzril laoreet ex ad, adipiscing nulla, et dolore, vel te in dolor te, feugait dolore ex vel erat duis. Ut diam commodo ad eu in consequat esse in ut wisi aliquip dolore feugiat wisi eum dignissim tincidunt vel, nostrud. Ut vulputate eum euismod, diam minim eros consequat lorem aliquam et ad luptatum illum sit suscipit ut, tation in dolore euismod et iusto nulla. Iusto wisi odio quis nisl feugiat adipiscing luptatum minim. Illum, quis, erat, dolore qui quis sit dolor veniam blandit ullamcorper ex, vero nonummy, duis exerci delenit ullamcorper at feugiat. Et, ullamcorper elit vulputate iusto esse luptatum duis autem esse nulla qui.

Praesent dolore et, delenit, laoreet dolore sed eros hendrerit consequat lobortis. Dolor nulla suscipit delenit commodo augue exerci magna veniam hendrerit dignissim duis ut feugait amet. Ad dolor suscipit iriure veniam blandit quis enim vulputate nulla facilisis volutpat vel in. Erat facilisis dolore ut veniam, duis facilisi wisi nulla aliquip vero praesent nibh molestie consectetuer nulla, iriure nibh exerci hendrerit. Vel, nostrud lobortis ut praesent dignissim tincidunt enim eum accumsan ea, nonummy duis. Ad autem feugait praesent, duis, accumsan tation enim facilisi qui te dolore magna velit, iusto esse eu, zzril vel enim zzril, te. Nisl illum, lobortis ut tation, elit luptatum ipsum, aliquam dolor sed minim consectetuer aliquip.

Tation exerci delenit ullamcorper at feugiat ullamcorper, ullamcorper elit vulputate iusto esse luptatum duis autem esse nulla qui. Volutpat praesent et at nisl ut in consequat blandit vel augue ut dolor illum facilisis zzril ipsum, augue odio, accumsan ea augue molestie lobortis zzril laoreet. Ex duis, te velit illum odio, nisl qui consequat aliquip qui blandit hendrerit. Ea dolor nonummy ullamcorper nulla lorem tation laoreet in ea, ullamcorper vel consequat zzril delenit quis dignissim, vulputate tincidunt ut."
%======================================================================
\chapter{Observations}
%======================================================================

This would be a good place for some figures and tables.

Some notes on figures and photographs\ldots

\begin{itemize}
\item A well-prepared PDF should be 
  \begin{enumerate}
    \item Of reasonable size, {\it i.e.} photos cropped and compressed.
    \item Scalable, to allow enlargment of text and drawings. 
  \end{enumerate} 
\item Photos must be bit maps, and so are not scaleable by definition. TIFF and
BMP are uncompressed formats, while JPEG is compressed. Most photos can be
compressed without losing their illustrative value.
\item Drawings that you make should be scalable vector graphics, \emph{not} 
bit maps. Some scalable vector file formats are: EPS, SVG, PNG, WMF. These can
all be converted into PNG or PDF, that pdflatex recognizes. Your drawing 
package can probably export to one of these formats directly. Otherwise, a 
common procedure is to print-to-file through a Postscript printer driver to 
create a PS file, then convert that to EPS (encapsulated PS, which has a 
bounding box to describe its exact size rather than a whole page). 
Programs such as GSView (a Ghostscript GUI) can create both EPS and PDF from PS files.
Appendix~\ref{AppendixA} shows how to generate properly sized Matlab plots and save them as PDF.
\item It's important to crop your photos and draw your figures to the size that
you want to appear in your thesis. Scaling photos with the 
includegraphics command will cause loss of resolution. And scaling down 
drawings may cause any text annotations to become too small.
\end{itemize}

For more information on \LaTeX\, see these  \href{https://uwaterloo.ca/information-systems-technology/services/electronic-thesis-preparation-and-submission-support/ethesis-guide/creating-pdf-version-your-thesis/creating-pdf-files-using-latex/latex-ethesis-and-large-documents}{course notes}. 
\footnote{
Note that while it is possible to include hyperlinks to external documents,
it is not wise to do so, since anything you can't control may change over time. 
It \emph{would} be appropriate and necessary to provide external links to 
additional resources that you provide for a multimedia ``enhanced'' thesis. 
But also note that if the \package{hyperref} package is not included, 
as for the print-optimized option in this thesis template, any \cmmd{href} 
commands in your logical document are no longer defined.
A work-around employed by this thesis template is to define a dummy \cmmd{href} 
command (which does nothing) in the preamble of the document, 
before the \package{hyperref} package is included. 
The dummy definition is then redifined by the
\package{hyperref} package when it is included.
}

The classic book by Leslie Lamport \cite{lamport.book}, author of \LaTeX , is worth a look too, and the many available add-on packages are described by 
Goossens \textit{et al} \cite{goossens.book}.

%======================================================================
\chapter{Introduction}
%======================================================================
In the beginning, there was $\pi$:

\begin{equation}
   e^{\pi i} + 1 = 0  \label{eqn_pi}
\end{equation}
A \gls{computer} could compute $\pi$ all day long. In fact, subsets of digits of $\pi$'s decimal approximation would make a good source for psuedo-random vectors, \gls{rvec} . 

%----------------------------------------------------------------------
\section{State of the Art}
%----------------------------------------------------------------------

See equation \ref{eqn_pi} on page \pageref{eqn_pi}.\footnote{A famous equation.}

\section{Some Meaningless Stuff}

The credo of the \gls{aaaaz} was, for several years, several paragraphs of gibberish, until the \gls{dingledorf} responsible for the \gls{aaaaz} Web site realized his mistake:

"Velit dolor illum facilisis zzril ipsum, augue odio, accumsan ea augue molestie lobortis zzril laoreet ex ad, adipiscing nulla. Veniam dolore, vel te in dolor te, feugait dolore ex vel erat duis nostrud diam commodo ad eu in consequat esse in ut wisi. Consectetuer dolore feugiat wisi eum dignissim tincidunt vel, nostrud, at vulputate eum euismod, diam minim eros consequat lorem aliquam et ad. Feugait illum sit suscipit ut, tation in dolore euismod et iusto nulla amet wisi odio quis nisl feugiat adipiscing luptatum minim nisl, quis, erat, dolore. Elit quis sit dolor veniam blandit ullamcorper ex, vero nonummy, duis exerci delenit ullamcorper at feugiat ullamcorper, ullamcorper elit vulputate iusto esse luptatum duis autem. Nulla nulla qui, te praesent et at nisl ut in consequat blandit vel augue ut.

Illum suscipit delenit commodo augue exerci magna veniam hendrerit dignissim duis ut feugait amet dolor dolor suscipit iriure veniam. Vel quis enim vulputate nulla facilisis volutpat vel in, suscipit facilisis dolore ut veniam, duis facilisi wisi nulla aliquip vero praesent nibh molestie consectetuer nulla. Wisi nibh exerci hendrerit consequat, nostrud lobortis ut praesent dignissim tincidunt enim eum accumsan. Lorem, nonummy duis iriure autem feugait praesent, duis, accumsan tation enim facilisi qui te dolore magna velit, iusto esse eu, zzril. Feugiat enim zzril, te vel illum, lobortis ut tation, elit luptatum ipsum, aliquam dolor sed. Ex consectetuer aliquip in, tation delenit dignissim accumsan consequat, vero, et ad eu velit ut duis ea ea odio.

Vero qui, te praesent et at nisl ut in consequat blandit vel augue ut dolor illum facilisis zzril ipsum. Exerci odio, accumsan ea augue molestie lobortis zzril laoreet ex ad, adipiscing nulla, et dolore, vel te in dolor te, feugait dolore ex vel erat duis. Ut diam commodo ad eu in consequat esse in ut wisi aliquip dolore feugiat wisi eum dignissim tincidunt vel, nostrud. Ut vulputate eum euismod, diam minim eros consequat lorem aliquam et ad luptatum illum sit suscipit ut, tation in dolore euismod et iusto nulla. Iusto wisi odio quis nisl feugiat adipiscing luptatum minim. Illum, quis, erat, dolore qui quis sit dolor veniam blandit ullamcorper ex, vero nonummy, duis exerci delenit ullamcorper at feugiat. Et, ullamcorper elit vulputate iusto esse luptatum duis autem esse nulla qui.

Praesent dolore et, delenit, laoreet dolore sed eros hendrerit consequat lobortis. Dolor nulla suscipit delenit commodo augue exerci magna veniam hendrerit dignissim duis ut feugait amet. Ad dolor suscipit iriure veniam blandit quis enim vulputate nulla facilisis volutpat vel in. Erat facilisis dolore ut veniam, duis facilisi wisi nulla aliquip vero praesent nibh molestie consectetuer nulla, iriure nibh exerci hendrerit. Vel, nostrud lobortis ut praesent dignissim tincidunt enim eum accumsan ea, nonummy duis. Ad autem feugait praesent, duis, accumsan tation enim facilisi qui te dolore magna velit, iusto esse eu, zzril vel enim zzril, te. Nisl illum, lobortis ut tation, elit luptatum ipsum, aliquam dolor sed minim consectetuer aliquip.

Tation exerci delenit ullamcorper at feugiat ullamcorper, ullamcorper elit vulputate iusto esse luptatum duis autem esse nulla qui. Volutpat praesent et at nisl ut in consequat blandit vel augue ut dolor illum facilisis zzril ipsum, augue odio, accumsan ea augue molestie lobortis zzril laoreet. Ex duis, te velit illum odio, nisl qui consequat aliquip qui blandit hendrerit. Ea dolor nonummy ullamcorper nulla lorem tation laoreet in ea, ullamcorper vel consequat zzril delenit quis dignissim, vulputate tincidunt ut."
%======================================================================
\chapter{Observations}
%======================================================================

This would be a good place for some figures and tables.

Some notes on figures and photographs\ldots

\begin{itemize}
\item A well-prepared PDF should be 
  \begin{enumerate}
    \item Of reasonable size, {\it i.e.} photos cropped and compressed.
    \item Scalable, to allow enlargment of text and drawings. 
  \end{enumerate} 
\item Photos must be bit maps, and so are not scaleable by definition. TIFF and
BMP are uncompressed formats, while JPEG is compressed. Most photos can be
compressed without losing their illustrative value.
\item Drawings that you make should be scalable vector graphics, \emph{not} 
bit maps. Some scalable vector file formats are: EPS, SVG, PNG, WMF. These can
all be converted into PNG or PDF, that pdflatex recognizes. Your drawing 
package can probably export to one of these formats directly. Otherwise, a 
common procedure is to print-to-file through a Postscript printer driver to 
create a PS file, then convert that to EPS (encapsulated PS, which has a 
bounding box to describe its exact size rather than a whole page). 
Programs such as GSView (a Ghostscript GUI) can create both EPS and PDF from PS files.
Appendix~\ref{AppendixA} shows how to generate properly sized Matlab plots and save them as PDF.
\item It's important to crop your photos and draw your figures to the size that
you want to appear in your thesis. Scaling photos with the 
includegraphics command will cause loss of resolution. And scaling down 
drawings may cause any text annotations to become too small.
\end{itemize}

For more information on \LaTeX\, see these  \href{https://uwaterloo.ca/information-systems-technology/services/electronic-thesis-preparation-and-submission-support/ethesis-guide/creating-pdf-version-your-thesis/creating-pdf-files-using-latex/latex-ethesis-and-large-documents}{course notes}. 
\footnote{
Note that while it is possible to include hyperlinks to external documents,
it is not wise to do so, since anything you can't control may change over time. 
It \emph{would} be appropriate and necessary to provide external links to 
additional resources that you provide for a multimedia ``enhanced'' thesis. 
But also note that if the \package{hyperref} package is not included, 
as for the print-optimized option in this thesis template, any \cmmd{href} 
commands in your logical document are no longer defined.
A work-around employed by this thesis template is to define a dummy \cmmd{href} 
command (which does nothing) in the preamble of the document, 
before the \package{hyperref} package is included. 
The dummy definition is then redifined by the
\package{hyperref} package when it is included.
}

The classic book by Leslie Lamport \cite{lamport.book}, author of \LaTeX , is worth a look too, and the many available add-on packages are described by 
Goossens \textit{et al} \cite{goossens.book}.


%----------------------------------------------------------------------
% END MATERIAL
% Bibliography, Appendices, Index, etc.
%----------------------------------------------------------------------

% Bibliography

% The following statement selects the style to use for references.  
% It controls the sort order of the entries in the bibliography and also the formatting for the in-text labels.
\bibliographystyle{plain}
% This specifies the location of the file containing the bibliographic information.  
% It assumes you're using BibTeX to manage your references (if not, why not?).
\cleardoublepage % This is needed if the "book" document class is used, to place the anchor in the correct page, because the bibliography will start on its own page.
% Use \clearpage instead if the document class uses the "oneside" argument
\phantomsection  % With hyperref package, enables hyperlinking from the table of contents to bibliography             
% The following statement causes the title "References" to be used for the bibliography section:
\renewcommand*{\bibname}{References}

% Add the References to the Table of Contents
\addcontentsline{toc}{chapter}{\textbf{References}}

\bibliography{thesis}
% Tip: You can create multiple .bib files to organize your references. 
% Just list them all in the \bibliogaphy command, separated by commas (no spaces).

% The following statement causes the specified references to be added to the bibliography even if they were not cited in the text. 
% The asterisk is a wildcard that causes all entries in the bibliographic database to be included (optional).
\nocite{*}
%----------------------------------------------------------------------

% Appendices

% The \appendix statement indicates the beginning of the appendices.
\appendix
% Add an un-numbered title page before the appendices and a line in the Table of Contents
\chapter*{APPENDICES}
\addcontentsline{toc}{chapter}{APPENDICES}
% Appendices are just more chapters, with different labeling (letters instead of numbers).
% \chapter[PDF Plots From Matlab]{Matlab Code for Making a PDF Plot}
\label{AppendixA}
% Tip 4: Example (above) of how to get a shorter chapter title for the Table of Contents 
%======================================================================
\section{Using the Graphical User Interface}
Properties of Matab plots can be adjusted from the plot window via a graphical interface. Under the Desktop menu in the Figure window, select the Property Editor. You may also want to check the Plot Browser and Figure Palette for more tools. To adjust properties of the axes, look under the Edit menu and select Axes Properties.

To set the figure size and to save as PDF or other file formats, click the Export Setup button in the figure Property Editor.

\section{From the Command Line} 
All figure properties can also be manipulated from the command line. Here's an example: 
\begin{verbatim}
x=[0:0.1:pi];
hold on % Plot multiple traces on one figure
plot(x,sin(x))
plot(x,cos(x),'--r')
plot(x,tan(x),'.-g')
title('Some Trig Functions Over 0 to \pi') % Note LaTeX markup!
legend('{\it sin}(x)','{\it cos}(x)','{\it tan}(x)')
hold off
set(gca,'Ylim',[-3 3]) % Adjust Y limits of "current axes"
set(gcf,'Units','inches') % Set figure size units of "current figure"
set(gcf,'Position',[0,0,6,4]) % Set figure width (6 in.) and height (4 in.)
cd n:\thesis\plots % Select where to save
print -dpdf plot.pdf % Save as PDF
\end{verbatim}

% GLOSSARIES (Lists of definitions, abbreviations, symbols, etc. provided by the glossaries-extra package)
% -----------------------------
\printglossaries
\cleardoublepage
\phantomsection		% allows hyperref to link to the correct page

%----------------------------------------------------------------------
\end{document} % end of logical document
