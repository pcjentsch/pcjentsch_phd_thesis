\documentclass{article}



\begin{document}


\section{}

We can define a contagious process as the dynamical description of something propagating through a population of hosts. 

% This general notion is probably very old, but in ecology there is the idea of a "contagious disturbance" introduced (as far as I can tell) by Peterson in his PhD thesis which I find helpful. Peterson defines a contagious disturbance as a disturbance where it's extent is determined endogenously as a function of the system, in contrast to a non-contagious disturbance whose extent is determined exogenously.

A great deal of things can be described as a contagious, or equivalently an infectious process: humans spreading throughout the earth, opinions spreading throughout a population, and beneficial alleles throughout the genome of a species.

Our current situation has made the importance of understanding contagious processes extremely clear to more people than ever before.

\section{}
Due to their complexity, many of these infectious systems are studied in isolation. However, coupling is often fundamental to the story. Take for example, the kuramoto model, which are coupled periodic oscillators that exhibit spontaneous synchronization if their coupling is strong enough. 

While humanity's impact on the environment is the subject of over a century of scholarship, and the impact of the environment on humanity has an even longer history, the treatment of humanity and our environment as a coupled system, one that feeds back into itself, is relatively new. 

There are many problems that are better addressed from this perspective, and in this presentation and thesis I hope to provide some examples of that. 

\section{}
Beginning with the first project. The background to this chapter is that we wanted to construct a model of Sars-CoV-2 spread that accounted for behavioural dynamics. 

Population sentiment towards the use of many non-pharmeceutical interventions can majorly affect the transmission rate, as we saw in some countries that were able to keep cases in the single or double digits with a combination of mask usage, contact tracing, and physical distancing.

An important aspect to COVID-19 infection, especially at the beginning of pandemic was it's effect on the elderly. Ontario long term care homes especially were severely affected, but the risk of symptoms, hospitalization and death increase dramatically with age. 

Age also affects how NPIs and lockdowns affect the population. For instance, School shutdowns and stay at home orders can dramatically decrease transmission in children, but not transmission in adults.  

The COVID-19 vaccine rollout has already occurred, but at this point in the pandemic it was far off on the horizon. The question was how to prioritize groups receiving the vaccine. Due to the aforementioned mortality in the elderly, age structuring the compartments seemed like a good way to approach that. 


\section{}

At the point in the pandemic where we conducted this research, there were many unknowns about how the vaccine rollout would work, and what the social and epidemiological landscape would look like when it happened, so we wanted to be able to answer this question with respect to a bunch of the relevant parameters.


For this chapter I will describe our model, briefly how we fit it to data, and then the results where we compare outcomes for different vaccinations strategies as a function of key parameters.



\section{}

To give an overview of the structure of our model, we used a compartmental impulsive differential equation model of the infection with compartments outlined here. This is coupled to a simple social model of the adoption of NPIs within the population.

The population goes through the following states: Susceptible, Vaccinated but still susceptible (since evidence has shown that most vaccines in development are not perfectly effective), vaccinated and immune, exposed, pre-symptomatic, infectious and asymptomatic, infectious and symptomatic, and recovered. These disease compartments are also age structured, that is each disease compartment is further divided into 16 compartments, where each compartment represents a five year age group, except the oldest compartment.

Each person in the population is also assumed to either use NPIs or not, and so we denote the proportion of people using NPIs by the state variable $x(t)$.



\section{}
In this project we incorporate feedback between the population and the infection, by modeling the adoption of NPIs in a game theoretical framework. Here is the payoff matrix between two "players", where each has the choice to use or not use NPIs. 

We see that we experience the lowest risk of infection when both players are using NPIs. However, it can be beneficial for one player to "defect", because they get the benefit of not social distancing while enjoying a lower risk of infection, than if nobody social distanced. In this sense they are free-riding off the protection from others.

Under this model, we are all playing this game, all the time with everyone. Therefore we model the use of NPIs as a population game, where each person is "playing" against the average behaviour of the population.

We couple the fraction of people using NPIs, $x(t)$,  to the disease dynamics by modifying the infection rate proportional to $x(t)$.

\section{}

In our analysis, we compare the reduction in  cumulative mortality after five years from the beginning of the pandemic, between each vaccination strategy.

The strategies are oldest first, youngest first, vaccinating everyone at the same rate, and a contact-based strategy, where the ratios are given by the normalized leading eigenvector of the contact matrices. A plot of this strategy is shown in the figure here. The "oldest first" strategy protect a vulnerable age group while the other three strategies are designed to interrupt transmission.


\section{}
Approximate bayesian computation was used estimate the posteriors of the parameters using case data from Ontario, Canada.

The proportion of people using NPIs, $x(t)$ was fit to google mobility data. They have a dataset that measures the percentage change in baseline of visits to recreational and retail destinations, and we use that as a proxy for $x(t)$.

Of course, workplaces cannot be completely moved to remote, so we used googles data for the percentage of workplaces filled under baseline to measure this.

We sampled 400 particles from the posterior distributions obtained through this process to perform the simulations used for the results.  

The coupled aspect of our model is important here as without the feedback of falling cases causing drops in transmission rates due to lower NPI usage, the data is difficult to fit, even with seasonality. This can be seen in appendix of Chapter 1 where we show the model fit without social distancing dynamics. 

\section{}


Ok now onto the results. This plot shows the percentage reduction in mortality, relative to no vaccination, for each of the four strategies as a function of the vaccination rate psi, expressed as a percentage of the total population per week. We considered two dates for the onset of vaccination, the first of January 2021, and the first of september. These correspond to the end dates of a two-dose course of vaccination. We are also assuming a baseline efficacy of 75\% against both disease and transmissibility.

We identify a few regimes for model dynamics from this plot. If the vaccine is available soon and the vaccination rate is not low (over 1\% per week), we see the the oldest first strategy reduces mortality the most. However, if the vaccination rate is about \%0.5 per week, then contact-based and other transmission-interrupting strategies are best.

If the vaccine is available later, then for low vaccination rates, the oldest first strategy is best, but for vaccination rates above 1.0   \%, transmission-interrupting strategies become optimal again. As the vaccination rate increases more, the strategies converge as everyone is either infected or vaccinated too quickly for there to be much difference.

% We see the youngest first strategy performs very poorly. The under-performance of the youngest first strategy occurs because in populations with strong age-assortative mixing, the indirect benefits of vaccination are “wasted" if vaccination is first concentrated in specific age groups before being extended to the rest of the population.

% The relative performance of the strategies in these three regimes is generally unchanged across the full range of values for the shutdown threshold.

For context, 
(10,350 vaccines per day currently)/(population of ontario) * 100 * 7 = \%0.507 per week 

\section{}

In the next two slides we show timeseries plots of the model regimes identified previously.


(a) timely vaccination prevents third wave

(b) partial vaccination and indirect protection help during the third wave

\section[Slide]{}

(c) partial vaccination and indirect protection help during the third wave

(d) slow and late vaccination fails to prevent third wave.

\section{}

We also performed a parameter plane sensitivity analysis over all combinations of shutdown threshold capital T, and vaccination rate psi. This plot shows the strategy that reduces mortality the most over all realizations.Generally vaccinating vulnerable is the best strategy most often, and we see that the shutdown threshold does not affect that very much. 

Mention \%0.5 column loss
\section{}

To generate these histogram plots we took parameter plane from the previous slide, and grouped each of the solutions based on which strategy performed best for that parameter set. Then the groups were organized into a histogram based on the number of recovered people at vaccination start date. Vertical lines denote the median of the distribution in each subplot.

\section{}

So to sum up, in this chapter we develop a compartmental population model of the virus transmission and vaccination, coupled to a simple population-level social model. 


We showed that there are times when transmission interrupting strategies can be more effective, depending on the level of pre-existing immunity, or recovered people. 

Broadly, more pre-existing natural immunity causes transmission interrupting strategies to be more effective.

This is only the main results we focused on, the remainder are covered in the thesis, including varying vaccine efficacy and also introducing a second social equation to account for vaccine refusal dynamics.

There are also a few things that I want to point out as well that we did not account for. The efficacy of the oldest first strategy definitely has a lot to do with the high mortality rate of older people, particular those over 80, and in particular Ontario had some very severe outbreaks in Long-term-care homes.

We are assuming that immunity does not wane at all, and that vaccination rate is constant, which is a big assumption as we have seen in Ontario for instance.



\section{}

My next project was this model of invasive forest pests spreading throughout Eastern Canada. In addition to Professor Bauch and Professor Anand, Dr. Denys Yemshanov, from the Great Lakes Forestry Centre, also helped supervise my work on this project. 

Invasive forest pests cause a ton of damage, both ecological and financial, to woodlots and forests each year. We have good evidence which shows that many of these pests, including the extremely destructive Emerald Ash Borer, can be transported very long distances in untreated firewood. This long distance transport greatly increases the rate at which pests can spread throughout the landscape.

Education and awareness campaigns are one of the major tools to prevent the spread of untreated firewood. 


\section{}

Our research questions in writing this chapter were to ask, which methods are effective in reducing forest pest spread? \\
\vspace{0.5cm}
When is each method most effective with respect to pest attributes? 

To address these questions we adapted a metapopulation model by Barlow et al. to work over a larger network and include additional possibilities for interventions.

We used this model to compare three strategies for reducing the spread of a pest as a function of pest-specific information, over a given period of time.

Since we wanted to include things like education and awareness campaigns in our model, we needed to use a coupled model with social dynamics. 

\section{}

The model studied, shown here, was a compartmental metapopulation model where each forest patch is modeled as a well-mixed population of hosts through which the pest spreads. This is coupled to a replicator equation which expresses the dynamics for the proportion of people in a given patch who buy firewood locally. Firewood is moved between patches at a rate proportional to the weight on the corresponding travel network edge, and from there the pest spread within a patch according to the infestation rate $A$. The parameters $I_a$ and $\gamma$ create a small threshold effect, so the level of pest needed for a positive growth rate needs to be larger when pest population is small. 

Parameters were estimated by trying to match aspects of epidemic curves from the EAB infestation, but there is a lot of uncertainty in this model specification especially in the behavior dynamics. 

\section{}

We will use the variable big-$T(t)$ as a major metric of performance. This variable measures the cumulative tree infections up until time little-t, defined as in eqn 5.

\section{}

This is a plot of the network data we use to weight the between-patch firewood import terms. The data, obtained from park reservation systems by the canadian forest service, shows the rough frequency of recreational park use trips.

\section{}

Here we show that generally it is tricky to get effective long-term protection with interception of firewood or social incentives. There is an interesting threshold effect, primarily in $U$. In the paper we conclude that this threshold is unattainable. While that is certainly true for $C_e$ as infested firewood is extremely difficult to intercept, I do not think it is obvious for $U$ since we have no real benchmark of the effect of education policy on $U$.

% In fact, based on timeseries estimates, $U = 4$ is probably achieveable, given that corresponds to only a  


\section{}

Many results in this chapter are expressed as changes in the marginal benefit of U. That is, fix $t_0$, and given $T(t_0)$ as a function of $U$ we construct a simple linear approximation around these points, and plot the resulting slope. This gives us a linear approximation in the marginal benefit of $U$, as shown in the inset plot. The justification for this is primarily that we are interested in how changes in $U$ will affect outcomes, not in it's value. From my experiments, this linear approximation is mostly reflective of the trend, but it is certainly capable of hiding interesting nonlinearities in the dynamics that might affect outcomes.  

This plot shows that small $f$, which controls the influence of infestation on transport strategy, can dramatically hinder public outreach.

\section{}

Again we look at marginal returns in U, and show that it becomes ineffective with large enough infestation rates. With low enough infestation rates however, it is quite effective. I suspect we would see this line shift up with an increase in the social learning rate $\sigma$.

\section{}

Here I am plotting the marginal returns in $U$ with respect to both inter and intra path infestation rates. As you might expect the story is that fast spread rapidly outpaces the ability of the population to react to it fast enough. 

\section{}

Lastly we discuss the comparison to patch quarantine. This is huge topic, but for our implementation we took isolated some number of nodes with the highest betweenness centrality, that is the nodes with the most shortest paths going through them, from the rest of the network for time delta-t, starting one year after detection of the pest.

Again we see a strong threshold effect, but we really need a massive, long quarantine to see lasting effects after 20 years. This is of course assuming there are no leaky patches, which is definitely not the case in practice.

\section{}

The conclusion that we draw from this model is that fire-wood interception methods are not effective, and that education and quarantine are marginally effective in the prevention of pest spread, depending on specific pest parameters. I think we have seen that not just with EAB but with all number of invasive species, transmission limiting methods are just not going to be very effective because there are always leaks and a few leaks is often all it takes. 

\section{}

In contrast to the previous chapter, which discussed invasive species. Here we study an endemic pest and disturbance system, Mountain Pine Beetle and wildfire. 

This was my first research project as a graduate student, initially it was the subject of my masters until I transferred to the PhD program. We spent quite a long time reworking it in revisions, partially due to work on the COVID-19 model, but also there were some issues with the model that had to be worked out. It was not published when the thesis was submitted but it is now.  

The background is that these two disturbances, while natural features of dry pine forests in the central cordillera of North America, are generally more prevalent than in the 19th and 20th century. This is at least partially due to a warming climate, but there has been some scholarship that suggests that modern silviculture is also at fault. 


\section{}

The MPB model I chose to begin was the model of Duncan and coauthors, partially for its dynamical simplicity, and I was very interested in discrete time dynamics at that point.

The key aspect of MPB dynamics that this, and most other models, incorporate is that young trees are not susceptible. Our model assumes a lower bound for susceptibility at 50 years of age. These forests tend to have very large soil seed banks, so new trees are assumed to occupy any open canopy space.  

We extended this model with a simple model of the number of trees burned in year. I will discuss this more on the equations slide.
\section{}

This is a plot of model states.

explain snags

note weird indexing

$I_n$ is the trees that will be infested in after the spring of year $n$

$F_n$ is the trees that are burnt snags before the spring of year $n$, so they were burned in year $n-1$.




\section{}

I will do a quick overview of the model equations.

This model conserves forest floor space. This equation determines new trees coming up in the place of trees that have previously burned or been infested. We also assume Juveniles die at some rate d per year. 

Susceptibles in the next year will be the susceptibles from the current year, plus the oldest juveniles, minus the trees which are burned, infested, or both.


$P_n$ is the fire severity, which we assume is proportional to the total land area $T$, minus the total number of trees burned from preceding years, decaying by an exponential term controlled by $\kappa$.

This $max$ function is not present in the chapter, and for the parameter ranges we consider it is not needed, but I think it should be included.

We add noise to both the infested and burnt trees compartments as well.


\section{}

Increasing the burning rates of juvenile and susceptible classes reduces MPB outbreaks almost completely, although they still exist as a small endemic population.

It also stabilizes fire dynamics to reduce fire years, as seen in panel d).

\section{}

Here we see timeseries of the model dynamics with and without the aforementioned trimming protocol, known as FTP. FTP works by cutting down the $m$ largest juvenile age classes by the fraction $tau$.

The top panel is a snapshot of the juvenile age distribution at a fixed time, the other panels are timeseries of the model states. You can see that FTP really flattens the age distribution, and at a glance the size of wildfires each year is not much higher, while MPB outbreaks are noticeably smaller on average.  


\section{}

Here we make this conclusion solid by averaging outbreak sizes over the ranges of $\alpha_1$ and $\alpha_2$, respectively. In panel a) that the percent reduction in MPB size is consistent through most of the space, in panel b) we show that it is still effective if conducted every 5 years. We also tested a analog of controlled burning, and it is effective if burning rates are low but wildly ineffective if they are larger. Note also that this we are removing only $15$ of the top 8 juvenile classes, which is not very much, although doing this with precision in practice is probably not possible. 

\section{}


This method of modeling fire beetle systems is able to match observations made by ecologists about heterogeneity in stand structure with a simple model of MPB and wildfire. 

The stand thinning process is an effective way to increase resilience of the forest to MPB outbreaks. 

In retrospect, the MPB attack threshold is not as sharp as we model it, and it tends to decrease as a function of the MPB density. Larger populations will attack and kill smaller trees, and even less desirable hosts. In addition, it would have been interesting to model the fire dynamics within each summer as a continuous process.  
\section{Conclusion} 

We developed three simple models of coupled human-environment systems. We show in each chapter how the dynamical feedback obtained from the coupling let us answer novel questions about the systems.
% \begin{itemize}
%     \item Chapter 1: developed a disease-behaviour model of COVID-19 to address questions about vaccine prioritization
%     \item Chapter 2: we used a socio-ecological model of forest pest spread to compare the efficacy of measures to prevent invasive pest spread
%     \item Chapter 3: we created a model of coupled MPB and wildfire dynamics to shed light on the stand dynamics of this ecosystem 
% \end{itemize}
\section{}

\end{document}