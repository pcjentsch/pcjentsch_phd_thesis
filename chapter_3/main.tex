\%======================================================================
\chapter{Prioritising COVID-19 vaccination in changing social and epidemiological landscapes: a mathematical modelling study}
\%======================================================================

\begin{abstract}
  During the COVID-19 pandemic, authorities must decide which groups to prioritise for vaccination in a shifting social–epidemiological landscape in which the success of large-scale non-pharmaceutical interventions requires broad social acceptance. We aimed to compare projected COVID-19 mortality under four different strategies for the prioritisation of SARS-CoV-2 vaccines. We developed a coupled social–epidemiological model of SARS-CoV-2 transmission in which social and epidemiological dynamics interact with one another. We modelled how population adherence to non-pharmaceutical interventions responds to case incidence. In the model, schools and workplaces are also closed and reopened on the basis of reported cases. The model was parameterised with data on COVID-19 cases and mortality, SARS-CoV-2 seroprevalence, population mobility, and demography from Ontario, Canada (population 14.5 million). Disease progression parameters came from the SARS-CoV-2 epidemiological literature. We assumed a vaccine with 75\% efficacy against disease and transmissibility. We compared vaccinating those aged 60 years and older first (oldest-first strategy), vaccinating those younger than 20 years first (youngest-first strategy), vaccinating uniformly by age (uniform strategy), and a novel contact-based strategy. The latter three strategies interrupt transmission, whereas the first targets a vulnerable group to reduce disease. Vaccination rates ranged from 0.5\% to 5\% of the population per week, beginning on either Jan 1 or Sept 1, 2021. Case notifications, non-pharmaceutical intervention adherence, and lockdown undergo successive waves that interact with the timing of the vaccine programme to determine the relative effectiveness of the four strategies. Transmission-interrupting strategies become relatively more effective with time as herd immunity builds. The model predicts that, in the absence of vaccination, 72000 deaths (95\% credible interval 40000–122000) would occur in Ontario from Jan 1, 2021, to March 14, 2025, and at a vaccination rate of 1.5\% of the population per week, the oldest-first strategy would reduce COVID-19 mortality by 90.8\% on average (followed by 89.5\% in the uniform, 88.9\% in the contact-based, and 88.2\% in the youngest-first strategies). 60000 deaths (31000–108000) would occur from Sept 1, 2021, to March 14, 2025, in the absence of vaccination, and the contact-based strategy would reduce COVID-19 mortality by 92.6\% on average (followed by 92.1\% in the uniform, 91.0\% in the oldest-first, and 88.3\% in the youngest-first strategies) at a vaccination rate of 1.5\% of the population per week.Interpretation The most effective vaccination strategy for reducing mortality due to COVID-19 depends on the time course of the pandemic in the population. For later vaccination start dates, use of SARS-CoV-2 vaccines to interrupt transmission might prevent more deaths than prioritising vulnerable age groups.

  
\end{abstract}


\section{Introduction}

The COVID-19 pandemic has imposed a massive global health burden as waves of infection move through populations around the world1.  Both empirical analyses and mathematical models conclude that non-pharmaceutical interventions (NPIs) are effective in reducing COVID-19 case incidence2–4.  However, pharmaceutical interventions are highly desirable given the socio-economic costs of lockdown and physical distancing. Dozens of vaccines are  in development5, and model-based analyses are exploring the question of which groups should get the COVID-19 vaccine first6–8.  

When vaccines become available, we will face a very different epidemiological landscape from the early pandemic9. Many populations will already have experienced one or more waves of COVID-19.  As a result of natural immunity, the effective reproduction number Reff (the average number of secondary infections produced per infected person) will be reduced from its original value of approximately R0 = 2.2 in the absence of pre-existing immunity10. Epidemiological theory tells us that as Reff (or R0) decline toward 1, the indirect benefits of transmission-blocking vaccines become stronger.  For instance, if R0 ≈ 1.5, such as for seasonal influenza, only an estimated 33\% percent of the population needs immunity for transmission to die out in a homogeneously mixing population11,12.  This effect was evidenced by the strong suppression of influenza incidence in Australia in Spring 2020 due to NPIs targeted against COVID-19.13

This effect has stimulated a literature comparing the vaccination of groups that are responsible for most transmission to vaccination of groups that are vulnerable to serious complications from the infection. Natural immunity to SARS-CoV-2 will likely continue to rise in many populations on account of further infection waves. Given these likely changes to the epidemiological landscape before the vaccine becomes available, we suggest this question is worthy of investigation in the context of COVID-19. 

The social landscape will also look very different when vaccines become available. This aspect is crucial to understanding the pandemic. Scalable non-pharmaceutical interventions (NPIs) like physical distancing, hand-washing and masks are often one of the few available interventions when a novel pathogen emerges.  Flattening the COVID-19 epidemic curve was possible due to a sufficient response by populations willing to adhere to public health recommendations.  Therefore, pandemic waves are not simply imposed on populations--they are a creation of the population response to the pathogen. They exemplify coupled social-epidemiological systems where disease dynamics and behavioural dynamics interact with one another14. 

Approaches to modelling coupled social-epidemiological dynamics vary15–19.  Some models have used evolutionary game theory to model this two-way feedback in a variety of coupled human-environment systems14,20–25.  Evolutionary game theory captures how individuals learn social behaviours from others while weighing risks and benefits of different choices. In this framework, individuals who do not adopt NPIs can “free-ride” on the benefits of reduced transmission generated by individuals who do adopt NPIs15. 

Here, our objective is to compare projected COVID-19 mortality under four strategies for the prioritisation of COVID-19 vaccines: older individuals first, children first, uniform allocation, and a novel strategy based on the contact structure of the population. We use an age-structured model of SARS-CoV-2 transmission, including evolutionary game theory to model population adherence to NPIs and changes to mobility patterns.  We use scenario and sensitivity analysis to identify how strategy effectiveness responds to possible changes in the social-epidemiological landscape that may occur before and after vaccines become available.  

\section{Model Overview}

Structure and parameterisation.  We developed an age-structured SEPAIR model (Susceptible, Exposed, Presymptomatic, Asymptomatic, Symptomatic, Removed) with ages in 5-year increments. Upon infection, individuals enter a latent period where they are infected but not yet infectious (“Exposed”).  After the latent period, individuals become presymptomatically infectious, and then either symptomatically or asymptomatically infectious, before finally entering the Removed compartment when their infectiousness ends. We did not model testing or contact tracing explicitly, although we assume infected individuals are ascertained at some rate. Transmission occurs through an age-specific contact matrix, susceptibility to infection is age-specific, and we include seasonality due to changes in the contact patterns throughout the year.  To infer model parameters, we fitted the model to Ontario COVID-19 case notifications stratified by age and time, Ontario seroprevalence data, and Ontario mobility data.  Use of seroprevalence data ensured that our estimates of transmission were not biased by case under-reporting. Remaining model parameter values were fixed using Ontario demographic and mortality data, and literature on COVID-19 serial interval and incubation periods.  Details of our model structure, parameterization, data sources, and model fits appear in the appendix, pp 1-11. 

Both schools and workplaces are closed when the number of ascertained active cases surpasses 50\%, 100\%, 150\%, 200\%, or 250\% of the peak ascertained active cases that occurred during the first wave (the “shutdown threshold”, T), and are re-opened again when cases fall below that threshold. Individuals interact with other individuals at a specified rate and switch between adherence and non-adherence to NPIs, including mobility restrictions, by comparing the cost of practicing NPIs against the cost of not practicing NPIs and thereby being subject to an increased risk of infection according to the prevalence of ascertained cases.  Both school and workplace closure and population level of adherence to NPIs reduce transmission according to a specified efficacy (see Appendix, pp 1-5). 

Vaccine scenarios. We considered two dates for the onset of vaccination: 1 March 2021 and 1 September 2021. These correspond to the end dates of a two-dose course of vaccination lasting two weeks. We assumed it was possible to vaccinate 0.5\%, 1.0\%, 1.5\%, 2.5\%, or 5.0\% of the population per week (the “vaccination rate”, ψ0).  Our baseline scenario assumed a vaccine with 75\% efficacy in all ages, against both infection and transmission.  

The “oldest first” strategy administers the vaccine to individuals 60 years of age or older, first.  After all individuals in this group are vaccinated, the vaccine is administered uniformly to other ages. The “youngest first” strategy is similar, except it administers the vaccine to individuals younger than 20 years of age first.  The “uniform” strategy administers vaccines to all age groups uniformly, from the very start. The “contact-based” strategy allocates vaccines according to the relative role played by different age groups in transmission. This tends to prioritise ages 15-19 primarily, 20-59 secondarily, and the least in older or younger ages (Appendix, pages 4, 12).  The ``oldest first" strategy targets a vulnerable age group while the other three strategies are designed to interrupt transmission.  We also explored an optimal strategy that optimizes age-specific vaccine coverage to minimize the number of deaths over five years (Appendix, page 4). We also report on sensitivity analyses in the Results section. 

Role of the funding source. The funder had no role in any aspect of the study or the decision to publish. All authors had full access to all the data in the study and had final responsibility for the decision to submit for publication.

\section{Results} 

The Google mobility data that we use as a proxy for adherence to NPIs closely mirrors the COVID-19 case notification data over the time period used for fitting (Figure 1, open orange circles).  Since a heightened perception of COVID-19 infection risk simulates the adoption of NPIs26, which in turn reduces SARS-CoV-2 transmission2,3, this exemplifies a coupled  social-epidemiological dynamic.  The mirroring may furthermore represent convergence between social and epidemiological dynamics, which has been predicted for strongly coupled systems27. Moreover, the fit of the social submodel to the mobility data is as good as the fit of the epidemic submodel to case notification data (Figure 1), despite the fact that our social model consists of significantly fewer equations and a similar number of parameters as our epidemiological model. This shows how modelling population behaviour during a pandemic can be accomplished with relatively simple models. 

The model predicts additional pandemic waves from Fall 2020 onward, not only with respect to COVID-19 cases but also population adherence to NPIs and periods of school and workplace closure (Figure 2). The impact of the four strategies on COVID-19 cases and deaths depends on when the vaccine becomes available and how quickly the population can get vaccinated. Broadly speaking, vaccinating 60+ year-olds first reduces mortality the most out of all four strategies if vaccination begins in March 2021, whereas the uniform or contact-based strategies reduce mortality the most if vaccination begins in September 2021, unless the vaccination rate is very small. More specifically, we identify three regimes for model dynamics. We explore them through plots of infection incidence over time (Figure 3); plots of the cumulative number of deaths under all four strategies, as they depend on the vaccination rate (Figure 4) and shutdown threshold (Appendix, pp. 16-17); and plots showing which of the four strategies is the most effective (in terms of reducing mortality) as a function of the shutdown threshold and the vaccination rate (Figure 5). 

In the first regime, vaccination starts soon and the vaccination rate is relatively high (March availability, vaccinating 1.5\% or more of the population per week). A third wave in Fall 2021/Winter 2022 is thereby prevented (Figure 3a and appendix, page 13).  In this regime, enough people are vaccinated sufficiently far in advance to prevent a third wave, therefore it does not matter which age group is vaccinated first. All four strategies have very similar effectiveness, although “oldest first” has a slight edge over the other strategies (Figure 4a, 5a). 

In the second regime, either vaccination starts soon but the vaccination rate is lower (March availability, 1\% or less vaccinated per week, Figure 3b and Figure 2), or vaccination starts later but the vaccination rate is high (September availability, vaccinating 1.5\% or more of the population per week, Figure 3c and appendix, page 14).  In this intermediate scenario, a sufficient proportion of the population is vaccinated for indirect protection from the vaccine to become important, but not enough individuals are vaccinated to completely prevent a third wave.  As a result, the uniform and contact-based strategies are significantly more effective than the 60+ first strategy, while the “youngest first” strategy does worst of all (Figure 4, 5).  The under-performance of the youngest first strategy occurs because in populations with strong age-assortative mixing28, the indirect benefits of vaccination are “wasted" if vaccination is first concentrated in specific age groups before being extended to the rest of the population.  The 60+ first strategy is less affected by this because the COVID-19 case fatality rate is high in this age group. 

In the third regime, vaccination starts late and the vaccination rate is low (September availability, 1\% or less vaccinated per week; Figure 3d and appendix, page 15). This scenario does not allow enough time for indirect protection from vaccination to become strong.  As a result, the oldest first strategy has significantly higher effectiveness than the other three strategies (Figure 4b, 5b).  Overall mortality is higher for all strategies, on account of the delayed rollout of the vaccine. 
  
The relative performance of the strategies observed in these three regimes is generally unchanged across the full range of values for the shutdown threshold (Appendix, pp. 16-17).  Some of our violin plots show a dominant lobe and a smaller secondary lobe, on account of the fact that different intervention settings can generate a different number or timing of pandemic waves.  The optimized strategy always does best, by definition (Appendix, pp. 16-17). But it can be instructive to study how the optimized strategy allocates vaccines among the age groups. The optimal vaccine strategy allocates vaccines mostly to the 25-44 age group and secondly to 70+, depending on the vaccination rate (Appendix, page 18). These patterns suggest that the optimal strategy includes transmission interruption as a mechanism.   

Frequency histograms across all stochastic model realizations showing what percentage of the population has natural immunity at the start of a vaccine program, when a particular strategy was shown to work best, illustrate the role of indirect protection (Figure 6). In simulations where the oldest first strategy did best, the percentage of the population with natural immunity tends to be relatively low. This is expected, since indirect protection from vaccines is weaker when few people have natural immunity upon which vaccine indirect protection can build.  But when the uniform or contact-based strategy does best, more simulations exhibit a high level of natural immunity at the start of vaccination.  We note that the variance in these histograms is high, however, which underscores the role of other factors in the model such as timing and interaction between social and epidemiological dynamics. Studying model predictions under variation in the basic reproduction number, R011, also illustrates the role of indirect protection. As R0 is increased from 1.5 to 2.5 we observe that the vaccine becomes less effective in reducing mortality across all strategies, as expected (Appendix, page 19). This occurs because when R0 is larger the indirect protection of vaccines is weaker11. As a result, the effectiveness of the “oldest first” strategy is less compromised by the increase in R0 than the other strategies, at least when vaccination starts in September. 

We also studied how the best strategy changes depending on vaccine efficacy ranging from 40-90\% in 60+ year-olds and in <60 year-olds (Appendix, pp. 20-21). The uniform or contact-based strategies were the most effective in these ranges, except when (a) vaccination starts in September at 1\%/week and efficacy in <60 year-olds is less than 70\%, and and (b) vaccination starts in March at 2.5\%/week and efficacy is greater in 60+ year-olds than in <60 year-olds. We note that (b) is unlikely since vaccine efficacy typically falls with age, and (a) is expected since this places the model in the third dynamical regime. 

We also modelled dynamics of vaccinating behaviour after vaccines become available (Appendix, pp 4, 22-25).  Due to lack of empirical data, we explored a wide range for the social learning rate and the relative cost of vaccination versus infection.  Either the uniform or contact-based strategies were most effective, except when the relative cost of the vaccine is very low, in which case oldest first is the best strategy (Appendix, pp 22). Vaccine refusal increases as the vaccine cost rises (Appendix, pp 23-25). Since vaccine refusal in the targeted age group forces vaccination of other age groups instead, it makes all strategies behave more like the uniform strategy, although age-specific behaviours could change these predictions. 

Our baseline inferred value of R0≈1.7 was lower than many published estimates10.  We ran simulations with R0=2.5 for December 2020 onward and found that “oldest first” was somewhat more effective across a broader region of parameter space for September availability, particularly at higher vaccination rates (Appendix, pp 26). Finally, we also ran simulations with 30\% higher and lower ascertainment for December 2020 onward to capture potential changes to COVID-19 testing and found that it had little impact on which strategy was most effective (Appendix, pp 27-28).  

\section{Discussion}

Our social-epidemiological model suggests that if a COVID-19 vaccine becomes available later in the pandemic, using SARS-CoV-2 vaccines to interrupt transmission might prevent more COVID-19 deaths than using the vaccines to target those 60+ years of age, depending on when the vaccine becomes available and how quickly the population can be vaccinated. These results are driven by the fact that the vaccine may only become available after populations have had one or more waves of immunizing infections. As a result, the effective reproduction number Reff could be significantly closer to 1 than the basic reproduction number R0 ≈ 2.2 that applies to susceptible populations. In this regime, vaccines have disproportionately large indirect protective effects11.  

Several studies have used compartmental models to study prioritisation of age groups for COVID-19 vaccination6–8. These models vary widely in terms of study populations, representation of population heterogeneity, interventions, and assumptions about when vaccination starts. Similar to our results, Matrajt et al8  find that the level of pre-existing immunity strongly dictates outcomes: when pre-existing immunity is high, the optimal strategy distributes the vaccine more evenly across age groups rather than prioritising older age groups. Buckner et al7 find that targeting 60+ year-olds is best for reducing mortality. They assumed that vaccination begins in December 2020, and they base initial conditions on case notifications in the United States in that month. Similarly, Bubar et al6 find that vaccinating 60+ year-olds works best for reducing mortality for vaccine programs starting in July 2020 in Belgium, or August 2020 in New York City. Our results agree with Refs. 6,7 for the scenario of March 2021 vaccine availability. However, we find it makes sense to switch to vaccinating other age groups by September 2021. Such a late vaccine start date was not analyzed in Refs. 6,7 although their findings might change if the models were re-initialized to accommodate vaccination starting in September 2021. 

Our analysis was limited by its focus on prioritisation of age groups.  We did not model other sources of heterogeneity such as geography, socio-economic status, sex, or race--all of which are important determinants of disease burden in this highly unequal pandemic. We did not model outbreaks in long-term care facilities, where the dynamics of transmission and indirect protection differ from the general population. Similarly, we did not distinguish healthcare or other essential workers. However, many of these individuals are working age adults, and thus vaccinating them first among other working adults is consistent with our uniform and contact-based strategies. Our mortality estimates assume ICU capacity is not exceeded. If ICU capacity is exceeded in the second wave, then our projected deaths will be an under-estimate, although we speculate that the relative performance of the four strategies would not change. We used a single population model, but inter-population mobility can influence transmission dynamics: a large influx of infectious persons from another population can weaken the indirect protection afforded by vaccines. 

We used changes to baseline time spent at retail and recreational outlets to represent population adherence to NPIs.  Such mobility data is an imperfect proxy for physical distancing and will not capture mask use or hand-washing.  We did not have high resolution mobility data on these practices, although in future it may be possible to infer information about these practices by combining information from phone surveys with online social media data. Our simple ascertainment process in the model was designed to implicitly capture the effects of COVID-19 PCR testing, contact tracing and isolation (TTI). But without explicitly representing them, it is impossible for us to study combined strategies of vaccination and TTI, or to anticipate how specific changes to TTI would influence our findings. 

Finally, the model was parameterised with data from Ontario, Canada. The projected impact of the four vaccine strategies may differ in settings with different epidemiological or social characteristics.  At the same time, we note that our findings rely upon a robust epidemiological effect that occurs when Reff becomes small. Therefore, the only thing that may change in other settings is the timing of the switch to vaccine strategies that interrupt transmission. 

We opted for a coupled social-epidemiological model on account of the importance of interactions between population behaviour and disease dynamics for the control of COVID-19 in the absence of preventive pharmaceutical interventions. Our model generated significantly different projections in our sensitivity analysis where population behaviour was assumed constant, which is similar to conventional approaches to transmission modelling. Our social model is less complicated than our epidemiological model and despite this, the coupled social-epidemiological model fitted population-level behaviour as readily as it fitted the epidemic curve. Predicting behaviour is fraught with uncertainty, but so is predicting an epidemic curve. Moreover, digital data on behaviour and sentiment that can be used to model social dynamics is increasingly available29.  Given this, we suggest a role for more widespread use of  social-epidemiological models during pandemics. 

To apply these results to COVID-19 pandemic mitigation, large-scale seroprevalence surveys before the onset of vaccination could ascertain the level of a population's natural immunity.  Age-structured compartmental models could be initialized with this information to generate population-specific projections. In populations where SARS-CoV-2 seropositivity is high due to a Fall/Winter 2020 wave, vaccinating to interrupt transmission may reduce COVID-19 mortality more effectively than targeting vulnerable groups. 
